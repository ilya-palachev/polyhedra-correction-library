\documentclass[usePics]{lomonosov}
 
\begin{thesis}

\Title{О методах восстановления геометрических тел по измерениям опорной функции
и их приложении к задаче восстановления тел по теневым
контурам}{{Палачев\,И.\,А.}}

\Author{Палачев~Илья~Александрович}{Аспирант}{Механико-математический
факультет МГУ имени М.\,В.\,Ломоносова}{Москва}{Россия}{palachev.ilya@yandex.ru}

Методы восстановления геометрических тел по измерениям опорной функции
разрабатывались в течение почти 30 лет для различных приложений: в
геометрической томографии [1, 5], обработке данных лазерного радара [2] и
магнитно-резонансной визуализации [4]. В данной работе показано, что данные
методы применимы для задачи восстановления тел по теневым контурам. Также
приводится подход, позволяющий существенно оптимизировать один из них.

\begin{definition}
Пусть $K$ --- выпуклое тело в $\mathbb{R}^{n}$, а $S^{n - 1}$ ---
единичная сфера в $\mathbb{R}^{n}$. Тогда \textbf{опорной функцией} тела $K$
называется функция, определенная на $S^{n - 1}$ следующим образом:
\begin{equation}
h_{K}(u) = \sup \limits	_{x \in K} (x, u)	
\end{equation}
\end{definition}

Как известно, всякое выпуклое тело однозначно задается своей опорной функцией.

\begin{definition}
Обозначим $(u_{1}, \ldots, u_{m})$ -- \textbf{опорные направления} (т. е. 
единичные направления), $(h_{1}, \ldots, h_{m})$ -- \textbf{опорные числа} (т.
е. измерения опорной функции тела $K$, полученные с погрешностью):
\begin{equation}
h_{i} = h_{K}(u_{i}) + \varepsilon_{i}
\end{equation}
Если существует такое выпуклое тело $K^{*}$, что $h_{K^{*}}(u_{i}) = h_{i}$, то
набор опорных чисел $(h_{1}, \ldots, h_{m})$ называется \textbf{согласованным}.
\end{definition}

Суть вышеупомянутых методов состоит в том, чтобы по несогласованному набору
опорных чисел получить ближайший к нему (в одной из метрик $L_{1}, L_{2},
L_{\infty}$) согласованный набор $(h_{1}^{*}, \ldots, h_{m}^{*})$ и по нему
восстановить тело просто с помощью пересечения полупространств.

В работах [1, 2, 3, 4] данная задача сводилась к задачам линейного или 
квадратичного программирования:

\begin{equation}
||h - h^{0}|| \to \inf \;\;\; s. t. \; C h >= 0,
\end{equation}

а в работе [5] --- к эквивалентной задаче в других переменных (координатах
точек касания):

\begin{equation}\label{PalachevProblem}
||\{(x_{i}, u_{i})\}_{i = 1}^{m} - h^{0}|| \to \inf \;\;\; s. t. \;
(x_{i}, u_{i}) >= (x_{j}, u_{i}), i \neq j
\end{equation}

Основным недостатком данного метода является то, что в $\mathbb{R}^{3}$ число 
ограничений является квадратичным. В настоящем докладе приводится метод, с 
помощью которого можно исключить большую часть ограничений как избыточные для 
задачи, сформулированной в метрике $L_{\infty}$.

\begin{references}
\Source \ENGLISH{Prince\,J.\,L., Willsky\,A.\,S. Reconstructing convex sets
        from support line measurements // IEEE Transactions on Pattern Analysis
        and Machine Intelligence, 12(4):377-389, 1990.}

\Source \ENGLISH{Lele\,A.\,S., Kulkarni\,S.\,R., and Willsky\,A.\,S.
        Convex-polygon estimation from support-line measurements and
        applications to target reconstruction from laser-radar data // Journal
        of the Optical Society America A, 9(10):1693-1714, Oct 1992.}

\Source \ENGLISH{Karl\,W.\,C., Kulkarni\,S.\,R., Verghese\,G.\,C.,
        Willsky\,A.\,S. Local tests for consistency of support hyperplane data
        // Journal of Mathematical Imaging and Vision, 6(2-3):249-267, 1996.}

\Source \ENGLISH{Gregor\,J., Rannou\,F.\,R. 3D Support Function Estimation
        and Application for Projection MRI // International Journal for
        Imaging Systems and Technology, 12:43-50, 2002.}

\Source \ENGLISH{Gardner\,R.\,J., Kiderlen\,M. A new algorithm for 3D
        reconstruction from support functions // IEEE Transactions on Pattern
        Analysis and Machine Intelligence, 31(3):556-562, 2009.}
\end{references}

\end{thesis}

 