% Данные тезисы были оформлены в соответствии со следующей инструкцией:
% http://smu.cs.msu.su/sites/default/files/attachments/instruction-thesis-lomonosov-2015.pdf
\documentclass{lomonosov}
 
\begin{thesis}

\Title{О методах восстановления геометрических тел по измерениям опорной функции
и их приложении к задаче восстановления тел по теневым
контурам}{{Палачев\,И.\,А.}}

\Author{Палачев~Илья~Александрович}{Аспирант}{Механико-математический
факультет МГУ имени М.\,В.\,Ломоносова}{Москва}{Россия}{palachev.ilya@yandex.ru}

Методы восстановления геометрических тел по измерениям опорной функции
разрабатывались в течение почти 30 лет для различных приложений: в
геометрической томографии [1, 5], обработке данных лазерного радара [2] и
магнитно-резонансной визуализации [4]. В данной работе показано, что данные
методы применимы для задачи восстановления тел по теневым контурам, которые
можно интерпретировать как набор измерений опорной функции тела. Также
приводится подход, позволяющий существенно оптимизировать один из этих методов.

Как известно, всякое выпуклое тело $K \subset \mathbb{R}^{n}$ однозначно
задается своей опорной функцией $h_{K}(u) = \sup \limits_{x \in K} (x, u)$,
определенной на единичной сфере $S^{n - 1} \subset \mathbb{R}^{n}$.

\begin{definition}
Обозначим $(u_{1}, \ldots, u_{m})$ -- \textbf{опорные направления} (т. е. 
единичные направления), $(h_{1}, \ldots, h_{m})$ -- \textbf{опорные числа} (т.
е. измерения опорной функции тела $K$, полученные с погрешностью):
\begin{equation}
h_{i} = h_{K}(u_{i}) + \varepsilon_{i}
\end{equation}
Если существует такое выпуклое тело $K^{*}$, что $h_{K^{*}}(u_{i}) = h_{i}$, то
набор опорных чисел $(h_{1}, \ldots, h_{m})$ называется \textbf{согласованным}.
\end{definition}

Суть вышеупомянутых методов состоит в том, чтобы по несогласованному набору
опорных чисел получить ближайший к нему (в одной из метрик $L_{1}, L_{2},
L_{\infty}$) согласованный набор $(h_{1}^{*}, \ldots, h_{m}^{*})$ и по нему
восстановить тело. В работах [1-5] данная задача сводилась к задачам линейного 
или квадратичного программирования. Основным недостатком данного метода является 
то, что в $\mathbb{R}^{3}$ число ограничений является квадратичным. В настоящем 
докладе приводится метод, с помощью которого можно исключить большую часть 
ограничений как избыточные для задачи, сформулированной в метрике $L_{\infty}$
в терминах точек касания [5]:

\begin{equation}\label{PalachevProblem}
\begin{split}
\varepsilon &\to \inf \\
s. t. \;\;\; (x_{i}, u_{i}) &\geq (x_{j}, u_{i}), \;\;
1 \leq i \neq j \leq m, \\
|(x_{i}, u_{i}) &- h^{0}_{i}| < \varepsilon, \;\;\;
i = 1, \ldots, m
\end{split}
\end{equation}

где $\varepsilon$ -- дополнительная переменная, а $x_{i} \in \mathbb{R}^{3}$ ---
точки касания.

\begin{theorem}\label{PalachevTheprem}
Пусть система ограничений  в задаче (\ref{PalachevProblem}) разрешима для 
некоторого $\varepsilon^{0}$. Обозначим многогранник
$K = \bigcap \limits_{i = 1}^{m}\{(x, u_{i}) \leq h^{0}_{i} +
\varepsilon^{0}\}$. Пусть $\delta$ - полярное преобразование двойственности.
Тогда если отрезок, соединяющий точки
$\delta(\{(x, u_{i}) = h^{0}_{i} - \varepsilon^{0}\})$ и
$\delta(\{(x, u_{j}) = h^{0}_{j} - \varepsilon^{0}\})$ пересекает $\delta(K)$,
то условия $(x_{i}, u_{i}) \geq (x_{j}, u_{i})$ и
$(x_{j}, u_{j}) \geq (x_{i}, u_{j})$ избыточны в системе ограничений
(\ref{PalachevProblem})
\end{theorem}

Данный метод был протестирован на опорных измерениях, полученных из 100 теневых
контуров. Теорема \ref{PalachevTheprem} позволяет отбросить 85-95\% условий в
зависимости от входных данных. Полученная задача была решена с использованием
различных производственных и научно-исследовательских пакетов линейного
программирования.

\begin{references}
\Source \ENGLISH{Prince\,J.\,L., Willsky\,A.\,S. Reconstructing convex sets
        from support line measurements // IEEE Transactions on Pattern Analysis
        and Machine Intelligence, 12(4):377-389, 1990.}

\Source \ENGLISH{Lele\,A.\,S., Kulkarni\,S.\,R., and Willsky\,A.\,S.
        Convex-polygon estimation from support-line measurements and
        applications to target reconstruction from laser-radar data // Journal
        of the Optical Society America A, 9(10):1693-1714, Oct 1992.}

\Source \ENGLISH{Karl\,W.\,C., Kulkarni\,S.\,R., Verghese\,G.\,C.,
        Willsky\,A.\,S. Local tests for consistency of support hyperplane data
        // Journal of Mathematical Imaging and Vision, 6(2-3):249-267, 1996.}

\Source \ENGLISH{Gregor\,J., Rannou\,F.\,R. 3D Support Function Estimation
        and Application for Projection MRI // International Journal for
        Imaging Systems and Technology, 12:43-50, 2002.}

\Source \ENGLISH{Gardner\,R.\,J., Kiderlen\,M. A new algorithm for 3D
        reconstruction from support functions // IEEE Transactions on Pattern
        Analysis and Machine Intelligence, 31(3):556-562, 2009.}
\end{references}

\end{thesis}