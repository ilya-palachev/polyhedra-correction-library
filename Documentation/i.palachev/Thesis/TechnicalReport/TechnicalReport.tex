
% Copyright (c) 2009-20134 Ilya Palachev <iliyapalachev@gmail.com>

% Declare the class of document: size of paper, size of font, and etc.
% Type of the document is "article".
\documentclass[a4paper, 12pt, titlepage]{article} 

% Package that enables setting the size of free spaces at the border of the 
% page with the command  \geometry :
\usepackage{geometry}
\geometry {
   left=3cm,
   right=1.5cm,
   top=2cm,
   bottom=2cm
}

\usepackage[utf8]{inputenc}

\usepackage[english,russian]{babel}

\usepackage{amsmath}

%\usepackage{cmap}

\usepackage{indentfirst}

\usepackage{a4wide,amssymb}

%\usepackage[pdftex]{graphicx}

%\usepackage{wrapfig}

%\linespread{1.3}               % полтора интервала. Если 1.6, то два интервала
\pagestyle{plain}               % номерует страницы

\usepackage{graphicx}
\renewcommand{\topfraction}{1}
\renewcommand{\textfraction}{0}

% Package that enables usage of theorems and definition designed in a 
% standard way.
% http://en.wikibooks.org/wiki/LaTeX/Theorems
\usepackage{amsthm}

% Create environment for smart definitions
\theoremstyle{definition}
\newtheorem{SmartDefinition}{Определение}

% Create environment for smart theorems
\theoremstyle{plain}
\newtheorem{SmartTheorem}{Теорема}

% Create environment for smart lemmas
\theoremstyle{plain}
\newtheorem{SmartLemma}{Лемма}

% The following code enables back references:
\usepackage{color} 
\definecolor{darkgreen}{rgb}{0,.5,0} 
\usepackage[unicode,colorlinks,filecolor=blue,citecolor=darkgreen,pagebackref]
{hyperref}

% The package that provides symbols like \Square:
\usepackage{wasysym}

%opening
\title{Опорные методы восстановления выпуклых тел и их обобщения для задачи
восстановления многогранников по теневым и бликовым контурам \\ Технический
отчет}
\author{Палачев Илья}

\begin{document}

\maketitle

\tableofcontents

%%%%%%%%%%%%%%%%%%%%%%%%%%%%%%%%%%%%%%%%%%%%%%%%%%%%%%%%%%%%%%%%%%%%%%%%%%%%%%%%

\section{Введение}

\subsection{Исходная практическая проблема восстановления моделей алмазов по
теневым и бликовым контурам}

\subsection{Описание технологии построения моделей алмазов}

\subsection{Погрешности в теневых контурах и их последствия}

\subsection{Можно ли корректировать теневые контуры?}

\subsection{Смежные проблемы в компьютерной томографии, магнитно-резонансной
визуализации и обработке данных лазерного радара}

\subsection{Обзор работы}

%%%%%%%%%%%%%%%%%%%%%%%%%%%%%%%%%%%%%%%%%%%%%%%%%%%%%%%%%%%%%%%%%%%%%%%%%%%%%%%%

\section{Преобразование двойственности и опорное представление
выпуклого тела (исторический обзор)}

\subsection{Понятие полярного преобразования двойственности}

\subsection{Пересечение многогранников и выпуклая оболочка многогранников --
двойственные задачи}

\subsection{Пересечение полупространств и выпуклая оболочка вершин -- 
двойственные задачи}

\subsection{Опорное и точечное зондирование -- двойственные задачи}

\subsection{Понятие опорной функции выпуклого тела}

\subsection{Свойства опорного представления}

%%%%%%%%%%%%%%%%%%%%%%%%%%%%%%%%%%%%%%%%%%%%%%%%%%%%%%%%%%%%%%%%%%%%%%%%%%%%%%%%

\section{Первичная и двойственная формулировка исходной проблемы}

\subsection{Первичная формулировка проблемы}

\subsection{Двойственная формулировка проблемы}

\subsection{Критерий согласованности теневых контуров}

\subsection{Обобщение для случая бликовых контуров}

\subsection{Образы одной вершины на разных контурах}

\subsection{Разрешение нарушений плоскостности при подвижке вершины}

%%%%%%%%%%%%%%%%%%%%%%%%%%%%%%%%%%%%%%%%%%%%%%%%%%%%%%%%%%%%%%%%%%%%%%%%%%%%%%%%

\section{Опорные методы и опорные задачи восстановления выпуклых
тел (исторический обзор)}

\subsection{Понятие опорной функции выпуклого тела}

\subsection{Опорные задачи}

\subsection{Двухмерные опорные методы}

\subsection{Трехмерный метод восстановления базового тела}

\subsection{Трехмерный метод вершинного восстановления тела}

\subsection{Двухмерный метод обнаружения краев}

%%%%%%%%%%%%%%%%%%%%%%%%%%%%%%%%%%%%%%%%%%%%%%%%%%%%%%%%%%%%%%%%%%%%%%%%%%%%%%%%

\section{Эффективный критерий согласованности опорных данных}

\subsection{Критерий выпуклости многогранника}

\subsection{Связь между локальной согласованностью опорных данных и выпуклостью
многогранника в ребре}

\subsection{Критерий согласованности опорных данных}

%%%%%%%%%%%%%%%%%%%%%%%%%%%%%%%%%%%%%%%%%%%%%%%%%%%%%%%%%%%%%%%%%%%%%%%%%%%%%%%%

\section{Различные способы сведения исходной проблемы к опорным задачам}

\subsection{Наивная интерпретация сторон контуров как опорных плоскостей}

\subsection{Кластеризация вершин теневых контуров}

\subsection{Предварительная кластеризация и сведение к опорной задаче
вершинного восстановления}

\subsection{Учет невыпуклости в теневых контурах}

%%%%%%%%%%%%%%%%%%%%%%%%%%%%%%%%%%%%%%%%%%%%%%%%%%%%%%%%%%%%%%%%%%%%%%%%%%%%%%%%


\newpage
\bibliographystyle{plain}
\bibliography{references}

\end{document}
