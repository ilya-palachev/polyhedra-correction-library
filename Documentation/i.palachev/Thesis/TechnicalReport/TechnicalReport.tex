
% Copyright (c) 2009-20134 Ilya Palachev <iliyapalachev@gmail.com>

% Declare the class of document: size of paper, size of font, and etc.
% Type of the document is "article".
\documentclass[a4paper, 12pt, titlepage]{article} 

% Package that enables setting the size of free spaces at the border of the 
% page with the command  \geometry :
\usepackage{geometry}
\geometry {
   left=3cm,
   right=1.5cm,
   top=2cm,
   bottom=2cm
}

\usepackage[utf8]{inputenc}

\usepackage[english,russian]{babel}

\usepackage{amsmath}

%\usepackage{cmap}

\usepackage{indentfirst}

\usepackage{a4wide,amssymb}

%\usepackage[pdftex]{graphicx}

%\usepackage{wrapfig}

%\linespread{1.3}               % полтора интервала. Если 1.6, то два интервала
\pagestyle{plain}               % номерует страницы

\usepackage{graphicx}
\renewcommand{\topfraction}{1}
\renewcommand{\textfraction}{0}

% Package that enables usage of theorems and definition designed in a 
% standard way.
% http://en.wikibooks.org/wiki/LaTeX/Theorems
\usepackage{amsthm}

% Create environment for smart definitions
\theoremstyle{definition}
\newtheorem{SmartDefinition}{Определение}

% Create environment for smart theorems
\theoremstyle{plain}
\newtheorem{SmartTheorem}{Теорема}

% Create environment for smart lemmas
\theoremstyle{plain}
\newtheorem{SmartLemma}{Лемма}

% The following code enables back references:
\usepackage{color} 
\definecolor{darkgreen}{rgb}{0,.5,0} 
\usepackage[unicode,colorlinks,filecolor=blue,citecolor=darkgreen,pagebackref]
{hyperref}

% The package that provides symbols like \Square:
\usepackage{wasysym}

%opening
\title{Опорные методы восстановления выпуклых тел и их обобщения для задачи
восстановления многогранников по теневым и бликовым контурам \\ Технический
отчет}
\author{Палачев Илья}

\begin{document}

\maketitle

\tableofcontents

%%%%%%%%%%%%%%%%%%%%%%%%%%%%%%%%%%%%%%%%%%%%%%%%%%%%%%%%%%%%%%%%%%%%%%%%%%%%%%%%

\section{Введение}

\subsection{Исходная практическая проблема восстановления моделей алмазов по
теневым и бликовым контурам}

\subsection{Описание технологии построения моделей алмазов}

\subsection{Погрешности в теневых контурах и их последствия}

\subsection{Можно ли корректировать теневые контуры?}

\subsection{Смежные проблемы в компьютерной томографии, магнитно-резонансной
визуализации и обработке данных лазерного радара}

\subsection{Обзор работы}

%%%%%%%%%%%%%%%%%%%%%%%%%%%%%%%%%%%%%%%%%%%%%%%%%%%%%%%%%%%%%%%%%%%%%%%%%%%%%%%%

\section{Основные понятия: преобразование двойственности и опорное представление
выпуклого тела}

Ключевыми понятиями, использующимися в данной работе для решения задачи
восстановления трехмерных тел являются понятие \textbf{преобразования
двойственности} и понятие \textbf{опорной функции выпуклого тела}. Первое из
них активно используется для решения задач вычислительной геометрии, поскольку
оно позволяет путем несложных преобразований над входными данными
переформулировать задачу так, чтобы понятия "точка" и "плоскость" поменялись
местами, что позволяет сводить новые задачи к таким, для которых уже имеются
известные методы решения. Примерами приложений преобразования двойственности
являются такие задачи как:

\begin{itemize}
 \item Нахождение пересечения многогранников.
 \item Нахождение треугольника минимальной площади с вершинами на заданном
множестве точек \cite{journals/BIT/ChazelleG1985}
 \item Разбиение множества точек на плоскости прямой на 2 класса, лежащих по
 разные стороны от прямой
\end{itemize}

и многие другие.

Второе понятие широко используется в такой области как \textbf{геометрическая
томография}. Суть подхода состоит в том, что любое выпуклое тело однозначно
определяется своей опорной функцией и задачу восстановления тела можно свести к
задаче восстановления опорной функции этого тела. В работе
\cite{journals/cviu/GhoshK98} производится анализ опорного представления и,
между прочим, исследуется его связь с преобразованием двойственности.

\subsection{Понятие полярного преобразования двойственности}

Термин преобразования двойственности имеет хождение в вычислительной геометрии,
где он нашел широкое применение. В проективной геометрии данное понятие имеет
сходство с понятием поляры.

\begin{SmartDefinition}
 Пусть $\mathbb{R}^{n}$ - евклидово пространство. \textbf{Преобразованием 
двойственности} называется отображение $\delta$, определенное на множествах
аффинных подпространств пространства $R^{n}$, не содержащих начало координат
$0 = (0, \ldots, 0)$, и действующее следующим образом:
\begin{itemize}
 \item Пусть $A \neq 0$ -- подпространство размерности $0$ (т. е. точка),
 $A = (a_{1}, a_{2}, \ldots, a_{n}) \neq 0$. Тогда ему ставится в
 соответствие аффинное гиперпространство $\delta(A)$, описываемое линейным
 уравнением $a_{1} x_{1} + a_{2} x_{2} + \ldots + a_{n} x_{n} = 1$.
 \item Пусть $L$ -- некоторое аффиное гиперпространство (т. е. 
 подпространство размерности $n - 1$), не содержащее начало координат. Тогда его
 можно описать с помощью уравнения
 $a_{1} x_{1} + a_{2} x_{2} + \ldots + a_{n} x_{n} = 1$, причем коэффициенты
 этого уравнения определяются однозначно. Тогда преобразование двойственности
 ставит в соответствие $L$ точку $\delta(L) = (a_{1}, a_{2}, \ldots, a_{n})$.
 \item Пусть $L$ -- подпространство размерности $k \neq 0, n - 1, n$. Тогда ему
 ставится в соответствие подпространство размерности $n - k - 1$, определяемое
 как $\delta(L) = \{\delta(M) \; | \; L \subset M, dim M = n - 1\}$.
\end{itemize}

\end{SmartDefinition}

%% TODO: Добавить утверждение о том, что преобразование двойственности сохраняет
%% инцидентность

\subsection{Пересечение многогранников и построение выпуклой оболочки --
двойственные задачи}

Свое распространение в вычислительной геометрии преобразование двойственности
получило благодаря статье Мюллера - Препараты \ref{Muller1978217}, в которой
описывается метод построения пересечения выпуклых многогранников. Он описан
также в классической монографии Препараты - Шеймоса "Введение в вычислительную
геометрию" \ref{Preparata:1985:CGI:4333}. Основная идея заключается в том, что
преобразование можно расширить на класс многогранных тел, если сопоставлять
вершинам -- грани, граням -- вершины, а всякому ребру, соединяющему некоторые
две точки $A_{1}$ и $A_{2}$, -- ребро, разделяющее грани $\delta(A_{1})$ и
$\delta(A_{2})$. Справедлива также следующая теорема:

%% TODO: Разъяснить понятия полиэдра, многогранника и многогранного множества.
%% Как они связаны между собой?

\begin{SmartTheorem}
 Если $P$ -- выпуклый полиэдр, содержащий начало координат, то таким же является
 и двойственный к нему полиэдр $P^{(\delta)}$.
\end{SmartTheorem}

Метод построения пересечения можно сформулировать в виде следующей теоремы:

\begin{SmartTheorem}
 Пусть $P$ и $Q$ -- полиэдры, имеющие общую точку. Предположим без ограничения
 общности, что эта точка -- начало координат. Тогда справедливо следующее
 соотношение:
 \begin{equation}
  \delta(P \bigcap Q) = conv (P^{(\delta)} \bigcup Q^{(\delta)})
 \end{equation}
\end{SmartTheorem}

Здесь через $conv (P^{(\delta)} \bigcup Q^{(\delta)})$ обозначена выпуклая
оболочка объединения многогранников $P^{(\delta)}$ и $Q^{(\delta)}$, которую
можно найти как выпуклую оболочку объединения множеств $V_{P}^{(\delta)}$ и
$V_{Q}^{(\delta)}$ вершин полиэдров $P^{(\delta)}$ и $Q^{(\delta)}$:

\begin{equation}
 conv (P^{(\delta)} \bigcup Q^{(\delta)}) = conv (V_{P}^{(\delta)} \bigcup
 V_{Q}^{(\delta)})
\end{equation}

Выпуклую оболочку точек можно построить с помощью известных алгоритмов
построения выпуклых оболочек, имеющих сложность $O(N \log N)$

Данный результат можно обобщить и на случай бесконечных выпуклых многогранных
множеств. В частности, справедлива следующая теорема:

\begin{SmartTheorem}
 Пусть $\pi_{1}, \pi_{2}, \ldots, \pi_{n}$ -- набор произвольных плоскостей в
 $\mathbb{R}^{3}$, а $R_{\pi_{1}}, R_{\pi_{2}}, \ldots, R_{\pi_{n}}$ --
 соответствующие им полупространства, содержащие начало координат $O$. Тогда
 пересечение этих полупространств можно найти с помощью следующего соотношения:
 \begin{equation}
  \delta \left(\bigcap \limits_{i = 1}^{n} R_{\pi_{i}} \right) =
  conv (\delta(\pi_{1}), \delta(\pi_{2}), \ldots, \delta(\pi_{n}))
 \end{equation}
\end{SmartTheorem}

\subsection{Понятие опорной функции выпуклого тела}

Вторым ключевым понятием, которое будет использоваться в данной работе,
является так называемая \textbf{опорная функция выпуклого тела}, или 
\text{опорное представление выпуклого тела}.

Для простоты будем рассматривать выпуклые тела в трехмерном пространстве
$\mathbb{R}^{3}$, содержащие в своей внутренности начало координат $O$.
Для некоторых понятий будем давать определения и формулировки для случая
произвольной конечной размерности.

Как известно, выпуклое тело можно однозначно представить как пересечение 
полупространств всех его касательных плоскостей 
$K = \bigcap \limits_{x \in K} R_{x}$, где $R_{x}$ -- то из двух
полупространств, на которые плоскость $\pi_{x}$, касательная к телу $K$ в
точке $x$, делит $\mathbb{R}^{3}$, которое содержит в себе целиком все тело
$K$. Всякому выпуклому телу можно поставить в соответствие набор касательных 
плоскостей $\pi_{x}$, по которым его можно построить. Обратное неверно: не
всякому произвольному набору плоскостей можно поставить в соответствие тело,
касающееся всех этих плоскостей.

Всякую касательную плоскость $\pi_{x}$ можно однозначно охарактеризовать
единичным вектором нормали $u_{x}$ и расстоянием $h_{x}$ от начала координат 
$O$ до плоскости. Поскольку две разные касательные плоскости не могут иметь
одинаковые векторы нормалей, то можно рассматривать множество всех касательных
плоскостей выпуклого тела как функцию, определенную на всех единичных векторах
$u \in S_{2}$:

\begin{equation}h_{K}: S^{2} \to \mathbb{R}_{+}\end{equation}

Более общее понятие, включающее в себя выше указанное, было введено в 1903 году
Минковским.

\begin{SmartDefinition}
 \label{def:support-function}
 Будем называть \textbf{опорной функцией} выпуклого тела
 $K \subset \mathbb{R}^{n}$ следующую функцию
 $h_{K}: \mathbb{R}^{n} \to \mathbb{R}_{+}$:

 \begin{equation}h_{K}(u) = \max \limits_{x \in K}(x, u)\end{equation}
\end{SmartDefinition}

Если взять некоторую точку $u_{0}$ на единичной сфере $S^{n - 1}$, и вычислить
в ней значение опорной функции $h_{K}(u_{0})$, то по этим данным можно
построить касательную плоскость к выпуклому телу в некоторой (неизвестной!)
точке $x \in K$. Такую плоскость (в контексте, когда нет информации о положении
точки касания) принято называть опорной плоскостью.

\begin{SmartDefinition}
 \label{def:support-plane}
 \textbf{Опорной плоскостью} выпуклого тела $K$ по
 направлению $u \in S^{n - 1}$ называется плоскость с нормалью $u$, расстояние
 от которой до начала координат равно $h_{K}(u)$
\end{SmartDefinition}

Очевидно, что опорная функция выпуклого тела обладает следующим свойством:

\begin{equation}
 h_{K}(\lambda u) = \lambda h_{K}(u)
\end{equation}

Следовательно, для практики достаточно иметь дело только с ограничением опорной
функции на единичную сферу. В статье \cite{journals/jmiv/KarlKVW96} вводится
понятие \textbf{приведенной опорной функции}:

\begin{SmartDefinition}
 \label{def:support-plane}
 \textbf{Приведенной опорной функцией} выпуклого тела $K$ называется следующая
функция:
 \begin{equation}
 H_{K} (u) = h_{K} \left(\frac{u}{||u||}\right)
 \end{equation}
\end{SmartDefinition}

которая в действительности представляет собой расстояние от начала координат
$\mathbb{O}$ до опорной гиперплоскости по направлению $u$.

Более подробно свойства опорной функции рассматриваются в статье
\cite{journals/cviu/GhoshK98}.

\subsection{Опорное и точечное зондирование -- двойственные задачи}

Еще одно применение преобразования двойственности было открыто в диссертации
Greschak \cite{thesis/Greschak1985}. В ней исследовались в частности две задачи
восстановления и была доказана их двойственность (т. е. сведение одной задачи к
другой с помощью преобразования двойственности и наоборот). Приведем здесь
коротко этот результат.

Определим так называемую \textbf{пробную функцию} некоторого выпуклого тела $K$:

\begin{equation}
 probe(x) = \alpha x, \;\;\; \alpha x \in \partial K
\end{equation}

В работе Greschak формулируются следующие две задачи:

\begin{flushleft}
 (\textbf{Задача 1}) Восстановить неизвестный ограниченный выпуклый
 многогранник $K$ путем выделения последовательности точек
 $x_{1}, x_{2}, \ldots, x_{k}$ и вычисления пробной функции $probe(x)$ тела $K$
 в каждой из этих точек.
\end{flushleft}

\begin{flushleft}
 (\textbf{Задача 2}) Восстановить неизвестный ограниченный выпуклый
 многогранник $K$ путем выделения последовательности плоскостей
 $r_{1}, r_{2}, \ldots, r_{k}$ и вычисления опорной функции $support(x)$ тела
 $K$ в каждой из этих плоскостей.
\end{flushleft}

При этом в диссертации опорная функция определяется на плоскостях, а не на
векторах, то есть $support(\pi) = \alpha \pi$, где $\alpha \pi$ -- плоскость,
опорная к телу $K$ (то есть такая, что $\alpha \pi \cap K \neq \varnothing$ и
тело $K$ целиком лежит по одну сторону от плоскости $\alpha \pi$).

В работе представлены результаты, представляющие собой решения задач (1) и 
(2). Одним из результатов диссертации Greschak является следующая теорема:

\begin{SmartTheorem}
 Если алгоритм $A$ восстанавливает неизвестный ограниченный выпуклый 
 многогранник $P$ в $\mathbb{R}^{n}$ путем выделения последовательности точек
 $x_{1}, x_{2}, \ldots, x_{k}$ и вычисления пробной функции $probe(x)$ в каждой 
 из этих точек, то существует также и двойственный алгоритм $A^{*}$, который
 восстанавливает двойственный многогранник $P^{*}$ многогранника $P$ путем
 выделения последовательности гиперплоскостей $r_{1}, r_{2}, \ldots, r_{k}$
 (который являются двойственными к точкам $x_{1}, x_{2}, \ldots, x_{k}$)
 и вычисления пробной функции $probe(x)$ в каждой из этих гиперплоскостей.
 Обратное также верно.
\end{SmartTheorem}


%%%%%%%%%%%%%%%%%%%%%%%%%%%%%%%%%%%%%%%%%%%%%%%%%%%%%%%%%%%%%%%%%%%%%%%%%%%%%%%%

\section{Первичная и двойственная формулировка исходной проблемы}

Математическая формализация исследуемой задачи так или иначе должна быть
представлена как нахождение некоторого геометрического тела, которое бы
оптимальным образом соответствовало физическим измерениям, содержащим
погрешности. Любая формализация так или иначе должна представлять собой
математическую задачу, формулируемую в терминах точек, плоскостей, ребер,
многогранников и других геометрических объектов, решением которой будет являться
некоторый многогранник в $\mathbb{R}^{3}$.

Если взять эту формализацию, и заменить все слова "точка" на слова "плоскость",
а все слова "плоскость" на слова "точка", и так же поступить с со словами 
"содержит" и "содержится" (которые вообще можно заменить универсальным термином
"инцидентен / инцидентна"), то получится так называемая \textbf{двойственная
задача}.

Двойственная задача будет связана с исходной (которую будем называть
\mathbb{первичной задачей}) следующим образом. Пусть $\mathfrak{F}$ --
входные данные первичной задачи (точки и плоскости в $\mathbb{R}^{3}$ с связи
между ними) и пусть $P$ -- многогранник, который является решением первичной
задачи для входных данных $\mathfrak{F}$. Тогда многогранник
$\delta(\mathfrak{P})$ будет являться решением двойственной задачи для входных
данных $\delta(\mathfrak{F})$.

Этот факт позволяет решать первичную задачу следующим образом:
\begin{itemize}
 \item Из первичных входных $\mathfrak{F}$ данных получить двойственные им 
данные $\delta(\mathfrak{F})$
 \item Найти решение двойственной задачи для входных данных
$\delta(\mathfrak{F})$. Пусть оно представляет собой некоторый многогранник $K$.
 \item По многограннику $K$ построить двойственный ему многогранник $P =
 \delta(\mathfrak{K})$ -- решений первичной задачи для входных данных\
 $\mathfrak{F}$.
\end{itemize}

Итак, задачу восстановления выпуклого тела по его теневым контурам можно
сформулировать двумя способами. Поясним вышесказанное и покажем, какие свойства
проблемы и какие подходы к ее решению можно вывести из этого соображения.

\subsection{Первичная формулировка проблемы}

Первое, о чем следует сказать, есть то, что проблема возникает из прикладной
задачи, и поэтому допускает некоторую нечеткость в своей формулировке. Это дает
возможность предлагать различные математические формализации задачи.

\begin{SmartDefinition}
 Пусть $\pi$ -- некоторая плоскость в $\mathbb{R}^{3}$, проходящая через начало
 координат $O = (0, 0, 0)$. \textbf{Теневым контуром на плоскости $\pi$}
 называется всякая простая ломаная $C = A_{1} A_{2} \ldots A_{n}$ (т. е.
 ломаная без самопересечений) с конечным числом вершин, лежащая в плоскости
 $\pi$ таким образом, что ограниченная этой ломаной область в плоскости $\pi$
 содержит начало координат $O$.
\end{SmartDefinition}

\begin{SmartDefinition}
 Пусть $\nu = (\nu_{x}, \nu_{y}, \nu_{z}) \in \mathbb{R}^{3}$ -- некоторый
 вектор в $\mathbb{R}^{3}$ и пусть $\pi$ -- плоскость, задаваемая уравнением
 $\nu_{x} x + \nu_{y} y + \nu_{z} z = 0$, т. е. плоскость, проходящая через
 начало координат $O$ и имеющая нормаль $\nu$. \textbf{Теневым контуром по
 направлению $\nu$} называется всякий теневой контур на плоскости $\pi$.
\end{SmartDefinition}

\begin{SmartDefinition}
 Пусть $K$ - некоторое тело в $\mathbb{R}^{3}$ и пусть $\pi$ -- некоторая
 плоскость в $\mathbb{R}^{3}$. \textbf{Теневым контуром тела $K$ на плоскости
 $\pi$}, называется граница образа тела $K$ при отображении проецирования на
 плоскость $\pi$. Будем обозначать его через $C_{\pi}(K)$.
\end{SmartDefinition}

\begin{SmartDefinition}
 Пусть $K$ - некоторое тело в $\mathbb{R}^{3}$ и пусть $\nu = (\nu_{x},
 \nu_{y}, \nu_{z}) \in \mathbb{R}^{3}$ -- некоторый вектор в $\mathbb{R}^{3}$.
 \textbf{Теневым контуром тела $K$ по направлению  $nu$} называется теневой
 контур тела $K$ на плоскости $\pi$, задаваемой уравнением
 $\nu_{x} x + \nu_{y} y + \nu_{z} z = 0$, т. е. на плоскости, проходящей через
 начало координат $O$ и имеющей нормаль $\nu$. Будем обозначать его через
 $C_{\nu}(K)$
\end{SmartDefinition}

\begin{SmartDefinition}
 \textbf{Задачей восстановления трехмерного тела} называется конечный набор пар
 $\mathfrak{C} = \left\{(\nu_{i}, C_{i})\right}_{i = 1}^{m}$, где $\nu_{1},
 \nu_{2}, \ldots, \nu_{m}$ -- некоторый набор векторов в $\mathbb{R}^{3}$,
 называемых векторами проецирования, и $C_{i}$ есть некоторый теневой контур 
 по направлению $\nu_{i}$ для каждого $i = 1, 2, \ldots, m$.
\end{SmartDefinition}

\begin{SmartDefinition}
 Задача восстановления трехмерного тела
 $\mathfrak{C} = \left\{(\nu_{i}, C_{i})\right}_{i = 1}^{m}$ называется
 \textbb{тривиальной}, если существует такое трехмерное геометрическое тело $K$
 такое, что $C_{\nu_{i}}(K) = C_{i}$ для всех $i = 1, 2, \ldots, m$, т. е.
 теневым контуром тела $K$ по направлению $\nu_{i}$ является теневой контур
 $C_{i}$
\end{SmartDefinition}


\subsection{Двойственная формулировка проблемы}

\subsection{Критерий согласованности теневых контуров}

\subsection{Обобщение для случая бликовых контуров}

\subsection{Образы одной вершины на разных контурах}

\subsection{Разрешение нарушений плоскостности при подвижке вершины}

%%%%%%%%%%%%%%%%%%%%%%%%%%%%%%%%%%%%%%%%%%%%%%%%%%%%%%%%%%%%%%%%%%%%%%%%%%%%%%%%

\section{Опорные методы и опорные задачи восстановления выпуклых
тел (исторический обзор)}

\subsection{Понятие опорной функции выпуклого тела}

\subsection{Опорные задачи}

\subsection{Двухмерные опорные методы}

\subsection{Трехмерный метод восстановления базового тела}

\subsection{Трехмерный метод вершинного восстановления тела}

\subsection{Двухмерный метод обнаружения краев}

%%%%%%%%%%%%%%%%%%%%%%%%%%%%%%%%%%%%%%%%%%%%%%%%%%%%%%%%%%%%%%%%%%%%%%%%%%%%%%%%

\section{Эффективный критерий согласованности опорных данных}

\subsection{Критерий выпуклости многогранника}

\subsection{Связь между локальной согласованностью опорных данных и выпуклостью
многогранника в ребре}

\subsection{Критерий согласованности опорных данных}

%%%%%%%%%%%%%%%%%%%%%%%%%%%%%%%%%%%%%%%%%%%%%%%%%%%%%%%%%%%%%%%%%%%%%%%%%%%%%%%%

\section{Различные способы сведения исходной проблемы к опорным задачам}

\subsection{Наивная интерпретация сторон контуров как опорных плоскостей}

\subsection{Кластеризация вершин теневых контуров}

\subsection{Предварительная кластеризация и сведение к опорной задаче
вершинного восстановления}

\subsection{Учет невыпуклости в теневых контурах}

%%%%%%%%%%%%%%%%%%%%%%%%%%%%%%%%%%%%%%%%%%%%%%%%%%%%%%%%%%%%%%%%%%%%%%%%%%%%%%%%


\newpage
\bibliographystyle{plain}
\bibliography{references}

\end{document}
