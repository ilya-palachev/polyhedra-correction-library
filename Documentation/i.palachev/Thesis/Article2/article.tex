% Copyright (c) 2009-2015 Ilya Palachev <iliyapalachev@gmail.com>

\documentclass[a4paper, 10pt]{article}
%\usepackage[pass,paperwidth=17cm,paperheight=24cm]{geometry}

\usepackage[utf8]{inputenc}
\usepackage[english,russian]{babel}
\usepackage{graphicx}
\usepackage{wrapfig}
\usepackage{amsmath,amssymb}

% Package that enables usage of theorems and definition designed in a 
% standard way.
% http://en.wikibooks.org/wiki/LaTeX/Theorems
\usepackage{amsthm}

% Create environment for smart definitions
\theoremstyle{definition}
\newtheorem{SmartDefinition}{Определение}

% Create environment for smart theorems
\theoremstyle{plain}
\newtheorem{SmartTheorem}{Теорема}

% Create environment for smart lemmas
\theoremstyle{plain}
\newtheorem{SmartLemma}{Лемма}

\usepackage{caption,wasysym}
\usepackage{hyperref}

\title{БЫСТРЫЙ МЕТОД СОГЛАСОВАНИЯ ОПОРНЫХ ЧИСЕЛ}
\author{Палачев\,И.\,А.}
\date{20 июля 2015 г.}

\begin{document}
\maketitle

\begin{abstract}
Предложен новый алгоритм получения согласованного набора опорных чисел по
измерениям опорной функции выпуклого тела, позволяющий получить решение за
линейное время по числу измерений без использования линейного программирования.
Доказано обоснование применимости теории опорных чисел к задаче согласования
теневых контуров геометрических тел.
\end{abstract}

\textbf{Ключевые слова:} опорная функция, восстановление геометрических тел,
теневой контур, преобразование двойственности

\textbf{1. Введение.}
Данное исследование возникло из практической задачи восстановления трёхмерного
выпуклого тела по его теневым контурам. Подобная проблема возникает, например,
при построении моделей драгоценных камней в ювелирной промышленности, когда
тело сначала ставят на горизонтальную подставку и, вращая её, периодически
фотографируют камень на фоне светодиодной панели, в результате чего, после
обработки растровых снимков, получаются многоугольники -- теневые контуры.

В статье \cite{palachev} был предложен подход, который позволяет применить к
описанной выше задаче методы восстановления геометрических тел по измерениям их
опорных функций. Основная идея состоит в том, что всякий теневой контур является
непрерывным семейством измерений опорной функции тела на большом круге сферы.
Тем не менее, в данной статье не было обосновано, почему можно переходить от
непрерывного семейства опорных чисел к конечному набору: из всех опорных чисел
выбирался только конечный набор, который и подвергался затем процедуре
согласования.

Настоящая работа призвана восполнить этот недостаток: далее будет приведено
доказательство эквивалентности сильной и слабой согласованности теневых
контуров, которое служит обоснованием для применимости метода к рассматриваемой
практической задаче.

Следующим вопросом, который здесь будет рассмотрен, является метод быстрого
согласования опорных чисел, с помощью которого можно задачу в метрике
$L_{\infty}$ за линейное время, без использования линейного программирования.

\textbf{2. Основные понятия.}
Главным понятием, которое используется в данное статье, является понятие
\textit{опорной функции выпуклого тела}:

\begin{SmartDefinition}
 \label{def:support-function}
 \textbf{Опорной функцией} выпуклого тела $K \subset \mathbb{R}^{n}$
 называется
 $h_{K}: \mathbb{R}^{n} \to \mathbb{R}_{+}$:

 \begin{equation*}h_{K}(u) = \sup \limits_{x \in K}(x, u)\end{equation*}
\end{SmartDefinition}

В ряде практических задач, возникающих в компьютерной томографии
\cite{PrinceWillsky}, магнитно-резонансной визуализации
\cite{GregorRannou2001, GregorRannou2002} и обработке данных лазерного радара
\cite{LeleKulkarniWillsky}, встречаются такие, которые можно интерпретировать
как восстановление геометрической структуры выпуклого тела по измерениям его
опорной функции. А именно, речь идёт о следующей задаче: пусть имеется
некоторое реальное физическое тело $K$ и производятся эмпирические измерения
его опорной функции на конечном наборе единичных векторов
$u_{1}, u_{2}, \ldots, u_{m}$:

\begin{equation*}
 h_{i} = h_{K}(u_{i}) + \varepsilon_{i}, \;\; i = 1, 2, \ldots, m,
\end{equation*}

где $\varepsilon_{i}$ --- погрешность измерения величины $h_{i}$. Общепринятой
является следующая терминология: $h_{1}, h_{2}, \ldots, h_{m}$ суть
\textit{опорные числа} тела $K$ по \textit{опорным направлениям}
$u_{1}, u_{2}, \ldots, u_{m}$.

\begin{SmartDefinition}
 \label{def:consistency}
 Опорные числа $h_{1}, h_{2}, \ldots, h_{m}$ по направлениям \\
 $u_{1}, u_{2}, \ldots, u_{m}$ называются \textbf{согласованными}, если
 существует такое выпуклое тело  $K^{*} \subset \mathbb{R}^{n}$, что
 $h_{K^{*}}(u_{i}) = h_{i}$
\end{SmartDefinition}

Таким образом, поскольку не всякий набор действительных чисел является
согласованным набором опорных чисел некоторого выпуклого тела, естественным
образом возникает задача \textit{согласования опорных чисел}: по набору опорных
чисел $h^{0}_{1}, h^{0}_{2}, \ldots, h^{0}_{m}$ и опорным направлениям
$u_{1}, u_{2}, \ldots, u_{m}$ строится такой согласованный набор опорных чисел
$h^{*}_{1}, h^{*}_{2}, \ldots, h^{*}_{m}$, который реализует минимум функционала

\begin{equation*}
 I = || h - h^{0} ||_{X}
\end{equation*}

где $X$ -- одна из метрик  в $\mathbb{R}^{m}$, например,
$L_{2}, L_{1}, L_{\infty}$.

В работе \cite{GardnerKiderlen} задача согласования опорных чисел сводится к
следующей задаче математического программирования:

\begin{equation}
\label{equation:gardner-kiderlen}
 ||\{(x_{i}, u_{i})\}_{i = 1}^{m} - h^{0}||_{X} \to \inf \;\;\; s. t. \;\;
 (x_{i}, u_{i}) \geq (x_{j}, u_{i}), 1 \leq i \neq j \leq m
\end{equation}

где $x_{i} \in \mathbb{R}^{3}$ --- точки касания тела с опорной плоскостью,
имеющей нормаль $u_{i}$. В статье \cite{palachev} приводится метод, позволяющий
уменьшить число ограничений в задаче \ref{equation:gardner-kiderlen}.


\begin{thebibliography}{50}

\bibitem{palachev}
\emph{Палачев~И.А.}
Метод исключения избыточных ограничений в задаче восстановления тела по
измерениям его опорной функции //
Вычислительные методы и программирование.
2015.
\textbf{16}.
348--359.

\bibitem{PrinceWillsky}
\emph{Prince~J.L., Willsky~A.S.}
Reconstructing convex sets from support line measurements //
IEEE Transactions on Pattern Analysis and Machine Intelligence.
1990.
\textbf{12},
N 4.
377--389.

\bibitem{LeleKulkarniWillsky}
\emph{Lele~A.S., Kulkarni~S.R., Willsky~A.S.}
Convex-polygon estimation from support-line measurements and applications to
target reconstruction from laser-radar data //
Journal of the Optical Society of America A.
1992
\textbf{9},
N 10.
1693--1714.

\bibitem{GregorRannou2001}
\emph{Gregor~J., Rannou~F.R.}
Least-squares framework for projection {MRI} reconstruction //
Proceedings of SPIE.
2001.
\textbf{4322}.
888-898.

\bibitem{GregorRannou2002}
\emph{Gregor~J., Rannou~F.R.}
Three-dimensional support function estimation and application for projection
magnetic resonance imaging //
International Journal of Imaging Systems and Technology.
2002.
\textbf{12},
N 1.
43--50.

\bibitem{GardnerKiderlen}
\emph{Gardner~R.J., Kiderlen~M.}
A new algorithm for 3D reconstruction from support functions //
IEEE Transactions on Pattern Analysis and Machine Intelligence.
2009.
\textbf{31},
N 3.
556--562.

\end{thebibliography}

\end{document}