% Copyright (c) 2009-2015 Ilya Palachev <iliyapalachev@gmail.com>

\documentclass[a4paper, 11pt]{article}
%\usepackage[pass,paperwidth=17cm,paperheight=24cm]{geometry}

\usepackage[utf8]{inputenc}
\usepackage[english,russian]{babel}
\usepackage{graphicx}
%\usepackage{wrapfig}
\usepackage{amsmath,amssymb}

\usepackage{geometry}
\geometry{left=1.5cm}
\geometry{right=1.5cm}
\geometry{top=2cm}
\geometry{bottom=2cm}

% Package that enables usage of theorems and definition designed in a 
% standard way.
% http://en.wikibooks.org/wiki/LaTeX/Theorems
\usepackage{amsthm}

% Create environment for smart definitions


\newtheoremstyle{MyDefinitionStyle}%
{0}% space above
{0}% space below
{}% body font
{}% indent amount
{\bfseries}% head font
{.}% punctuation after head
{ }% space after head
{\indent\thmname{#1}\ \thmnumber{#2}}% head spec

\theoremstyle{MyDefinitionStyle}

\newtheorem{SmartDefinition}{Определение}

\newtheoremstyle{MyTheoremStyle}%
{0}% space above
{0}% space below
{\itshape}% body font
{}% indent amount
{\bfseries}% head font
{.}% punctuation after head
{ }% space after head
{\indent\thmname{#1}\ \thmnumber{#2}}% head spec
\theoremstyle{MyTheoremStyle}
\newtheorem{SmartTheorem}{Теорема}

\usepackage{caption,wasysym}
\usepackage{hyperref}

% For number, not asterisk in footnote for author:
\usepackage{titling}
\thanksmarkseries{arabic}

\title{\hbox{\normalsize УДК 519.6}
\textsc{Метод согласования измерений опорной функции выпуклого тела в метрике
$L_{\infty}$}}
\author{И.А.~Палачев\thanks{Палачев Илья Александрович --- соискатель каф.
вычислительной математики мех.--мат. ф-та МГУ, e-mail: palachev.ilya@yandex.ru}}
% Remove date from the title
\date{\vspace{-4\baselineskip}}

\makeatletter
\renewcommand\@biblabel[1]{#1.}
\makeatother

\begin{document}
% Remove word `Аннотация' from the title
\renewcommand{\abstractname}{\vspace{-\baselineskip}}

\maketitle

\begin{abstract}
Предложен новый алгоритм решения задачи согласования измерений опорной функции
выпуклого тела в метрике $L_{\infty}$, позволяющий получить решение за
квадратичное время по числу измерений без использования линейного
программирования.
Также доказана оценка скорости сходимости, которая имеет место при довольно
слабых условиях на входные данные, что дает возможность применить метод к более
широкому классу задач, чем это было прежде.
Разработанный алгоритм обладает большей  и гарантированной
стабильностью и предсказуемостью, нежели прочие алгоритмы, существовавшие и
применявшиеся для согласования измерений опорной функции.
Приведены детали реализации алгоритма и результаты его тестирования.
\end{abstract}

\emph{Ключевые слова:} опорная функция, восстановление геометрических тел,
теневой контур, преобразование двойственности

\begin{abstract}
A new algorithm is suggested for the estimation of convex body support function
measurements in $L_{\infty}$ metric, which allows to obtain the solution in
quadratic time (respect to the number of measurements) without the usage of
linear programming.
The speed of convergence is proved to be stable for a quite weak conditions on
the input data. This fact makes the algorithm workable for the the wider class
of problems than used to be before.
The implemented algorithm is guaranteed to be stable and predictable, unlike
the other support function estimation algorithms that existed and used to be
tried.
Implementation details and testing results are presented.
\end{abstract}

\emph{Keywords:} support function, geometric body reconstruction,
shadow contour, duality transformation

\textbf{1. Введение.}
Настоящее исследование возникло из практической задачи восстановления трехмерного
выпуклого тела по измерениям его опорной функции. Данная задача рассматривалась
в связи с разнообразными приложениями: в компьютерной томографии
\cite{PrinceWillsky}, при обработке данных лазерного радара
\cite{LeleKulkarniWillsky}, в магнитно-резонансной визуализации
\cite{GregorRannou2001, GregorRannou2002}, при построении трехмерных моделей
драгоценных камней в ювелирной промышленности~\cite{palachev}. Все эти
прикладные задачи так или иначе сводятся к решению так называемой
\textbf{задачи согласования измерений опорной функции выпуклого тела}
(разъяснение этих понятий см. ниже).

Во всех работах, посвященных решению задачи согласования, проблема сводилась к
решению задачи квадратичного (для метрики $L_2$) или линейного (для метрик
$L_1$ и $L_\infty$) программирования следующего вида:

\begin{equation*}
 ||h - h^0||_X \to \min, \;\;\;\;\; C h \ge 0,
\end{equation*}
где $h^0$ --- исходные значения измерений опорной функции, $h$ --- неизвестные
(переменные задачи), которые требуется найти, а $C$ --- линейная матрица.

Основным камнем преткновения при решении данной задачи для измерений опорных
функций тел в $\mathbb{R}^3$ являлся тот факт, что матрица $C$ содержит в себе
квадратичное ($O(N^2)$, где $N$ --- размерность вектора $h$) число строк.

Поэтому главным направлением исследования данной задачи в ряде работ было
упрощение условий согласованности, что позволяло записать задачу в более
компактном виде и, следовательно, решить ее за гораздо меньшее время. Так, в
работе~\cite{GardnerKiderlen} была найдена возможность записи задачи
программирования в более удобных терминах точек касания (до того времени само
нахождение условий было нетривиальной задачей).

Похожую цель преследовала и статья~\cite{palachev}, основной результат которой
дал возможность на большом классе входных данных избавиться примерно от 80\% условий
(последняя оценка, впрочем, была выведена экспериментально). Тем не
менее это позволило лишь сдвинуть границу применимости метода, но так и не
перевело его на уровень, достаточный, чтобы применить его к любым входным
данным.

Следующей сложностью на пути решения задачи стала довольно типичная для
вычислительных методов ситуация: не известен алгоритм, который бы
гарантированно работал для любых входных данных. Так, в мире существуют
как коммерческие (лишь некоторые из них доступны по академической лицензии),
так и свободные программные пакеты для решения задач квадратичного или
линейного программирования.
Коммерческие пакеты, часто не раскрывающие особенности своей реализации (или
оглашающие их лишь частично), не всегда могут удовлетворить пользователя,
желающего иметь гарантированное поведение программного пакета.

Свободные программные пакеты, такие, как, например, Ipopt~\cite{Ipopt}, который
как показано в статье~\cite{palachev}, довольно эффективно вычисляет решение
данной задачи, имеют открытый исходный код и основаны на результатах
научных статей~\cite{WachterBiegler, WachterBiegler2}, доказывающих
свойства реализованных в них алгоритмов. Тем не менее, метод внутренней точки,
являющийся основой реализации этого пакета, при всех его достоинствах
имеет такой недостаток,
что может впасть в режим восстановления, при котором существующие оценки
скорости сходимости довольно скудны~\cite{WachterBiegler2}.
Иначе говоря, алгоритм при наличии ряда неблагоприятных обстоятельств
может работать сколь угодно долго (если не ограничить число итераций) или
может вообще не сойтись.

В связи с этим возникает вопрос: \textbf{можно ли переформулировать задачу
согласования опорных чисел таким образом, чтобы алгоритм решения имел
гарантированную приемлемую скорость сходимости при любых входных данных?}

Настоящая работа призвана представить такой алгоритм для задачи в метрике
$L_\infty$. Метод позволяет свести задачу к алгоритму двоичного
поиска на отрезке действительных чисел, на каждом шаге которого применяются
алгоритмы, имеющие гарантированную вычислительную сложность (построение
выпуклой оболочки, задача обнаружений пересечения отрезка и многогранника,
линейные преобразования).
Более того, как будет показано далее, алгоритм при любых входных данных
имеет гарантированную (квадратичную) скорость сходимости,
какую метод внутренней точки позволял иметь лишь при выполнении большого ряда
ограничений на входные данные.

Также представлены результаты реализации и тестирования алгоритма,
подтверждающие теоретически доказанные оценки и демонстирующие приемлемую
производительность на ряде примеров, полученных из эмпирических данных
для восстановления тела по его теневым контурам.

\textbf{2. Основные понятия.} Приведем далее формализацию задачи согласования
измерений опорной функции выпуклого тела.
\begin{SmartDefinition}
 \label{def:support-function}
 \emph{Опорной функцией} выпуклого тела $K \subset \mathbb{R}^{n}$
 называется функция $h_{K}: \mathbb{R}^{n} \to \mathbb{R}_{+}$:

 \begin{equation*}h_{K}(u) = \sup \limits_{x \in K}(x, u),\end{equation*}
где $u$ --- произвольный вектор в $\mathbb{R}^n$,
$(x, u)$ --- евклидово скалярное произведение этого вектора на некоторую точку
$x \in K$.
\end{SmartDefinition}

В перечисленных выше работах рассматривается следующая задача. Пусть
имеется некоторое реальное физическое тело $K$ и производятся эмпирические
измерения его опорной функции $h_{1}, h_{2}, \ldots, h_{m}$ (опорные числа) на
конечном наборе единичных векторов $u_{1}, u_{2}, \ldots, u_{m}$
(опорные направления):

\begin{equation*}
 h_{i} = h_{K}(u_{i}) + \varepsilon_{i}, \;\; i = 1, 2, \ldots, m,
\end{equation*}
где $\varepsilon_{i}$ --- погрешности измерения опорных чисел $h_{i}$.

\begin{SmartDefinition}
 \label{def:consistency}
 Опорные числа $h_{1}, h_{2}, \ldots, h_{m}$ по направлениям
 $u_{1}, u_{2}, \ldots, u_{m}$ называются \emph{согласованными}, если
 существует такое выпуклое тело  $K^{*} \subset \mathbb{R}^{n}$, что
 $h_{K^{*}}(u_{i}) = h_{i}$.
\end{SmartDefinition}

Поскольку не всякий набор действительных чисел является
согласованным набором опорных чисел некоторого выпуклого тела, естественным
образом возникает задача \textit{согласования опорных чисел}: по набору опорных
чисел $h^{0}_{1}, h^{0}_{2}, \ldots, h^{0}_{m}$ и опорным направлениям
$u_{1}, u_{2}, \ldots, u_{m}$ строится такой согласованный набор опорных чисел
$h^{*}_{1}, h^{*}_{2}, \ldots, h^{*}_{m}$, который реализует минимум функционала

\begin{equation*}
 I = || h - h^{0} ||_{X},
\end{equation*}
где $X$ -- одна из метрик  в $\mathbb{R}^{m}$ (например,
$L_{2}, L_{1}, L_{\infty}$).

В работе~\cite{GardnerKiderlen} задача согласования опорных чисел сводится к
следующей задаче математического программирования:

\begin{equation*}
 ||\{(x_{i}, u_{i})\}_{i = 1}^{m} - h^{0}||_{X} \to \inf, \;\;\;
 (x_{i}, u_{i}) \geq (x_{j}, u_{i}), 1 \leq i \neq j \leq m,
\end{equation*}
где $x_{i} \in \mathbb{R}^{3}$ --- точки касания тела с опорной плоскостью,
имеющей нормаль $u_{i}$.


\textbf{3. Быстрый алгоритм согласования.}
Основным результатом работы~\cite{palachev} является следующая
\begin{SmartTheorem}\label{theorem:exhaustive-conditions}
Рассмотрим следующую систему ограничений относительно переменных точек
$x_i \in \mathbb{R}^{3}, i = 1, \ldots, n$ и переменного вещественного числа
$\varepsilon > 0$:

\begin{equation}
\label{equation:all-constraints}
 (x_{i}, u_{i}) \geq (x_{i}, u_{j}), \;\;\;
 |(x_{i}, u_{i}) - h_{i}| \leq \varepsilon,
\end{equation}
где $u_i, i = 1, \ldots, n$, --- набор фиксированных опорных направлений,
нормированных в евклидовой метрике (т.е. $||u_i||_{2} = 1, i = 1, \ldots, n$),
$h_i, i = 1 , \ldots, n$, --- набор соответствующих им фиксированных опорных
чисел.

Пусть система ограничений \eqref{equation:all-constraints} разрешима
для некоторого $\varepsilon^{0}$. Введем многогранник $K$ по следующей формуле:

\begin{equation}
\label{equation:upper-bounds-polyhedron}
 K = \bigcap \limits_{i = 1}^{m}\{(x, u_{i}) \leq h_{i} + \varepsilon^{0}\}.
\end{equation}

Пусть $\delta$ --- полярное преобразование двойственности.
Тогда если отрезок, соединяющий точки
$\delta(\{(x, u_{i}) = h_{i} - \varepsilon^{0}\})$ и
$\delta(\{(x, u_{j}) = h_{j} - \varepsilon^{0}\})$, пересекает
$\delta(K)$, то условия $(x_{i}, u_{i}) \geq (x_{j}, u_{i})$ и
$(x_{j}, u_{j}) \geq (x_{i}, u_{j})$ избыточны в системе ограничений
\eqref{equation:all-constraints}.
\end{SmartTheorem}

Рассмотрим задачу согласования опорных чисел в метрике $L_{\infty}$:

\begin{equation}
\label{equation:infinity-problem}
 \varepsilon \to \inf, \;\;\; (x_{i}, u_{i}) \geq (x_{i}, u_{j}), \;\;\;
 |(x_{i}, u_{i}) - h_{i}| \leq \varepsilon
\end{equation}

По сути, данная задача линейного программирования состоит в нахождении такого
минимального $\varepsilon$, при котором система ограничений
\eqref{equation:all-constraints} является разрешимой.

\begin{SmartTheorem}
 Система ограничений \eqref{equation:all-constraints} разрешима тогда и только
 тогда, когда ни одна из точек
 $\delta(\{(x, u_{i}) = h_{i} - \varepsilon\})$ не лежит во внутренности
 многогранника $\delta(K)$, где $K$ -- многогранник, определяемый соотношением
 \eqref{equation:upper-bounds-polyhedron}.
\end{SmartTheorem}

\textbf{Доказательство.}
Точка $\delta(\{(x, u_{i}) = h_{i} - \varepsilon\})$ не лежит во внутренности
многогранника $\delta(K)$ тогда и только тогда, когда плоскость
$\{(x, u_{i}) = h_{i} - \varepsilon\}$ пересекает многогранник $K$ (по
одному из свойств преобразования двойственности). В данном случае можно положить
$ x_{i} = h_{K}(u_{i}) $, и тогда 

\begin{align*}
 (x_{i}, u_{i}) & \geq h_{i} - \varepsilon, \\
 (x_{i}, u_{j}) & \leq h_{j} + \varepsilon, \; j = 1, \ldots, m.
\end{align*}

Построим таким образом набор точек $(x_{1}, \ldots, x_{m})$. По определению
опорной функции условия $(x_{i}, u_{i}) \geq (x_{i}, u_{j})$ выполняются
автоматически.
$\square$

Полученный результат позволяет организовать процесс нахождения решения задачи
линейного программирования \eqref{equation:infinity-problem} с помощью
следующего алгоритма:

% We don't use "enumerate" here, because:
%  (1) It breaks identation,
%  (2) We're going to reference to the number of algorithm steps further.

1. По исходным опорным данным $u_{i}, h_{i}, i = 1, \ldots, m$,
 построить тело $K^{0}$ как пересечение полупространств:
 \begin{equation*}
 \label{equation:naive-body}
  K^{0} = \bigcap \limits_{i = 1}^{m} \{(u_{i}, x) \leq h_{i}\}.
 \end{equation*}

2. Пусть $A_{1}, \ldots A_{s}$ --- вершины тела $K^{0}$. Вычислить опорные
 числа тела $K^{0}$ по направлениям $u_{i}, i = 1, \ldots, m$, следующим образом:
 \begin{equation*}
  h^{0}_{i} = \max \limits_{j = 1, \ldots, s} (u_{i}, A_{j}).
 \end{equation*}

3. Вычислить $L_{\infty}$-расстояние от полученных опорных чисел до
 исходных:
 \begin{equation*}
  \varepsilon^{0} = \max \limits_{i = 1, \ldots, m} |h_{i} - h^{0}_{i}|.
 \end{equation*}
 Положить $\varepsilon_{-} = 0, \;\; \varepsilon_{+} = \varepsilon_{0}$.

4. Если $|\varepsilon_{+} - \varepsilon_{-}| < \Delta \varepsilon$, то
 остановить алгоритм, в противном случае положить
 $\varepsilon = \frac{\varepsilon_{-} + \varepsilon_{+}}{2}$ и перейти к
 следующему шагу.

5. Вычислить точки
 $\delta(\{(x, u_{i}) = h_{i} + \varepsilon\}), \; i = 1, \ldots, m$, и построить
 их динамическую выпуклую оболочку $D$.

6. Вычислить точки
 $\delta(\{(x, u_{i}) = h_{i} - \varepsilon\}), \; i = 1, \ldots, m$, и
 проверить, лежит ли хотя бы одна из них во внутренности $D$. Если хотя бы одна
 лежит, то изменить величину $\varepsilon_{-} = \varepsilon$, иначе изменить
 величину $\varepsilon_{+} = \varepsilon$. Перейти к шагу 4.


На шаге 4 величина $\Delta \varepsilon$ фиксирована, т.е. является
параметром алгоритма, определяющим точность вычисления минимального допустимого
$\varepsilon$. Количество итераций, необходимое для завершения алгоритма,
очевидно, равно $\log_{2} \frac{\varepsilon_{0}}{\Delta \varepsilon}$. На каждом
шаге алгоритма требуется $2m$ раз вычислять двойственные образы плоскостей (за
константное число операций), один раз строить динамическую выпуклую оболочку
$m$ точек (что требует $O(m^2)$ времени при использовании алгоритма
построения трехмерной триангуляции Делоне~\cite{Devillers}) и $m$ раз проверять
принадлежность точки построенному многограннику (что требует $O(\log m)$
времени для каждой из $m$ точек по отдельности~\cite{Devillers}).
Следовательно, трудоемкость работы алгоритма составляет

\begin{equation*}
 \log_{2} \frac{\varepsilon_{0}}{\Delta \varepsilon} \cdot O(m^2).
\end{equation*}

Поскольку в реальных задачах величины $\varepsilon_{0}$ и
$\Delta \varepsilon$ различаются не больше чем на несколько порядков, то
можно считать, что представленный алгоритм имеет квадратичную сложность.

Также представляется целесообразным проверить гипотезу, что при соблюдении
разумного предположения о равномерной распределенности опорных направлений
$u_i, i = 1, \ldots, n$, на сфере возможно построение триангуляции Делоне
за линейное время~\cite{Attali}, что позволило бы говорить о
линейно-логарифмической сложности алгоритма.

\textbf{4. Реализация и тестирование алгоритма}.
Алгоритм был реализован на основе библиотеки алгоритмов вычислительной
геометрии CGAL\footnote{Computational Geometry Algorithms Library.
\url{http://www.cgal.org}}.
Пересечение полупространств на шаге алгоритма 1 и выпуклая
оболочка на шаге 5 вычислялись на основе алгоритма построения трехмерной
триангуляции Делоне~\cite{Devillers}. Выбор в пользу данного алгоритма был
сделан поскольку он позволяет производить быструю многократную проверку
принадлежности точки выпуклой оболочке~\cite{Devillers}.

Тестирование было произведено на опорных числах, полученных эмпирическим путем
на основе измерений реального выпуклого тела (драгоценного камня).
В метрике $L_{\infty}$ алгоритм работает приблизительно в 300
раз быстрее, чем существующий алгоритм решения задачи линейного программирования,
основанный на методе внутренней точки~\cite{Ipopt}. К примеру, решение задачи на
2912 опорных числах было получено за 0,98 с, в то время как
предыдущий алгоритм позволял получить решение лишь за 300 с~\cite{palachev}
(на одних и тех же вычислительных ресурсах).

Безусловно, лишь основываясь на тестировании алгоритма, нельзя судить о превосходстве
или недостатке предлагаемого метода по отношению к предыдущему. Тем не менее
определенные выводы возможно сделать на основе теоретической оценки скорости
работы алгоритма, представленного в работе~\cite{palachev}.
По теореме 4.7 статьи~\cite{WachterBiegler}
фильтрационный метод внутренней точки с линейным поиском при выполнении ряда
условий сходится сверхлинейно к минимуму
задачи~\eqref{equation:infinity-problem}. Поскольку
задача~\eqref{equation:infinity-problem}
содержит квадратичное число ограничений
(см.~\cite{GardnerKiderlen,palachev}), на каждом шаге метода выполняются
вычисления общей трудоемкостью не менее $O(m^2)$, т.е. такой же, какую дает
предлагаемый метод. Между тем на практике оказалось, что по меньшей мере одно
условие теоремы 4.7 из~\cite{WachterBiegler} не выполняется в практических
примерах задачи~\eqref{equation:infinity-problem}, а именно
алгоритм не должен входить в фазу восстановления (когда производится
минимизация суммарного отклонения ограничений). На примере, рассмотренном
в~\cite{palachev}, алгоритм внутренней точки входит в фазу восстановления.
Следовательно, в данном случае мы имеем лишь сходимость, гарантируемую теоремой
2 статьи~\cite{WachterBiegler2}, в которой не приводятся оценки скорости
сходимости.

Таким образом, предлагаемый метод позволяет за квадратичное время
гарантированно получить решение задачи \eqref{equation:infinity-problem},
в отличие от метода внутренней точки для
скорости сходимости которого применительно к задаче
\eqref{equation:infinity-problem} содержательных оценок пока нет.

\small
\begin{thebibliography}{50}

\bibitem{PrinceWillsky}
\emph{Prince~J.L., Willsky~A.S.}
Reconstructing convex sets from support line measurements //
IEEE Trans. Pattern Anal. and Machine Intel.
1990.
\textbf{12},
N 4.
377--389.

\bibitem{LeleKulkarniWillsky}
\emph{Lele~A.S., Kulkarni~S.R., Willsky~A.S.}
Convex-polygon estimation from support-line measurements and applications to
target reconstruction from laser-radar data //
J. Opt. Soc. Amer. A.
1992
\textbf{9},
N 10.
1693--1714.

\bibitem{GregorRannou2001}
\emph{Gregor~J., Rannou~F.R.}
Least-squares framework for projection {MRI} reconstruction //
Proc. SPIE.
2001.
\textbf{4322}.
888-898.

\bibitem{GregorRannou2002}
\emph{Gregor~J., Rannou~F.R.}
Three-dimensional support function estimation and application for projection
magnetic resonance imaging //
Int. J. Imaging Systems and Technol.
2002.
\textbf{12},
N 1.
43--50.

\bibitem{palachev}
\emph{Палачев~И.А.}
Метод исключения избыточных ограничений в задаче восстановления тела по
измерениям его опорной функции //
Вычисл. методы и програм.
2015.
\textbf{16}.
348--359.

\bibitem{GardnerKiderlen}
\emph{Gardner~R.J., Kiderlen~M.}
A new algorithm for 3D reconstruction from support functions //
IEEE Trans. Pattern Anal. and Machine Intel.
2009.
\textbf{31},
N 3.
556--562.

\bibitem{Ipopt}
\emph{Wachter~A., Biegler~L.T.}
On the implementation of a primal-dual interior point filter line search
algorithm for large-scale nonlinear programming. //
Math. Program.
2006.
\textbf{106},
N 1.
25--57

\bibitem{WachterBiegler}
\emph{Wachter~A., Biegler.~L.T.}
Line search filter methods for nonlinear programming: Local convergence. //
SIAM J. Optim.
2005.
\textbf{16},
N 1.
32--48

\bibitem{WachterBiegler2}
\emph{Wachter~A., Biegler.~L.T.}
Line search filter methods for nonlinear programming: Motivation and global
convergence. //
SIAM J. Optim.
2005.
\textbf{16},
N 1.
1--31

\bibitem{Devillers}
\emph{Devillers~O.}
The Delaunay Hierarchy //
Int. J. Foundations of Computer Science.
2002.
\textbf{13}.
163--180.

\bibitem{Attali}
\emph{Attali~D., Boissonnat~J.-D.}
A linear bound on the complexity of the Delaunay triangulation of points on
polyhedral surfaces //
Discr. and Comp. Geom.
2004.
\textbf{31},
N 3.
369--384.

\end{thebibliography}

\end{document}