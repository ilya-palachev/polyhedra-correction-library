% Copyright (c) 2009-2015 Ilya Palachev <iliyapalachev@gmail.com>

\documentclass[a4paper, 10pt]{article}
%\usepackage[pass,paperwidth=17cm,paperheight=24cm]{geometry}

\usepackage[utf8]{inputenc}
\usepackage[english,russian]{babel}

\usepackage{hyperref}

\title{БЫСТРЫЙ МЕТОД СОГЛАСОВАНИЯ ОПОРНЫХ ЧИСЕЛ}
\author{Палачев\,И.\,А.}
\date{20 июля 2015 г.}

\begin{document}
\maketitle

\begin{abstract}
Предложен новый алгоритм получения согласованного набора опорных чисел по
измерениям опорной функции выпуклого тела, позволяющий получить решение за
линейное время по числу измерений без использования линейного программирования.
Доказано обоснование применимости теории опорных чисел к задаче согласования
теневых контуров геометрических тел.
\end{abstract}

\textbf{Ключевые слова:} опорная функция, восстановление геометрических тел,
теневой контур, преобразование двойственности

\textbf{1. Введение.}
Данное исследование возникло из практической задачи восстановления трёхмерного
выпуклого тела по его теневым контурам. Подобная проблема возникает, например,
при построении моделей драгоценных камней в ювелирной промышленности, когда
тело сначала ставят на горизонтальную подставку и, вращая её, периодически
фотографируют камень на фоне светодиодной панели, в результате чего, после
обработки растровых снимков, получаются многоугольники -- теневые контуры.

В статье \cite{palachev} был предложен подход, который позволяет применить к
описанной выше задаче методы восстановления геометрических тел по измерениям их
опорных функций. Основная идея состоит в том, что всякий теневой контур является
непрерывным семейством измерений опорной функции тела на большом круге сферы.
Тем не менее, в данной статье не было обосновано, почему можно переходить от
непрерывного семейства опорных чисел к конечному набору: из всех опорных чисел
выбирался только конечный набор, который и подвергался затем процедуре
согласования.

Настоящая работа призвана восполнить этот недостаток: далее будет приведено
доказательство эквивалентности сильной и слабой согласованности теневых
контуров, которое служит обоснованием для применимости метода к рассматриваемой
практической задаче.

Следующим вопросом, который здесь будет рассмотрен, является метод быстрого
согласования опорных чисел, с помощью которого можно задачу в метрике
$L_{\infty}$ за линейное время, без использования линейного программирования.



\begin{thebibliography}{50}

\bibitem{palachev}
\emph{Палачев~И.А.}
Метод исключения избыточных ограничений в задаче восстановления тела по
измерениям его опорной функции //
Вычислительные методы и программирование.
2015.
\textbf{16}.
348--359.

\end{thebibliography}

\end{document}