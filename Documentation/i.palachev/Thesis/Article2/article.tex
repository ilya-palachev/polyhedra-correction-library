% Copyright (c) 2009-2015 Ilya Palachev <iliyapalachev@gmail.com>

\documentclass[a4paper, 10pt]{article}
%\usepackage[pass,paperwidth=17cm,paperheight=24cm]{geometry}

\usepackage[utf8]{inputenc}
\usepackage[english,russian]{babel}
\usepackage{graphicx}
\usepackage{wrapfig}
\usepackage{amsmath,amssymb}

% Package that enables usage of theorems and definition designed in a 
% standard way.
% http://en.wikibooks.org/wiki/LaTeX/Theorems
\usepackage{amsthm}

% Create environment for smart definitions
\theoremstyle{definition}
\newtheorem{SmartDefinition}{Определение}

% Create environment for smart theorems
\theoremstyle{plain}
\newtheorem{SmartTheorem}{Теорема}

% Create environment for smart lemmas
\theoremstyle{plain}
\newtheorem{SmartLemma}{Лемма}

\usepackage{caption,wasysym}
\usepackage{hyperref}

\title{БЫСТРЫЙ МЕТОД СОГЛАСОВАНИЯ ОПОРНЫХ ЧИСЕЛ}
\author{Палачев\,И.\,А.}
\date{20 июля 2015 г.}

\begin{document}
\maketitle

\begin{abstract}
Предложен новый алгоритм получения согласованного набора опорных чисел по
измерениям опорной функции выпуклого тела, позволяющий получить решение за
линейно-логарифмическое время по числу измерений без использования линейного
программирования. Доказано обоснование применимости теории опорных чисел к
задаче согласования теневых контуров геометрических тел.
\end{abstract}

\textbf{Ключевые слова:} опорная функция, восстановление геометрических тел,
теневой контур, преобразование двойственности

\textbf{1. Введение.}
Данное исследование возникло из практической задачи восстановления трёхмерного
выпуклого тела по его теневым контурам. Подобная проблема возникает, например,
при построении моделей драгоценных камней в ювелирной промышленности, когда
тело сначала ставят на горизонтальную подставку и, вращая её, периодически
фотографируют камень на фоне светодиодной панели, в результате чего, после
обработки растровых снимков, получаются многоугольники -- теневые контуры.

В статье \cite{palachev} был предложен подход, который позволяет применить к
описанной выше задаче методы восстановления геометрических тел по измерениям их
опорных функций. Основная идея состоит в том, что всякий теневой контур является
непрерывным семейством измерений опорной функции тела на большом круге сферы.
Тем не менее, в данной статье не было обосновано, почему можно переходить от
непрерывного семейства опорных чисел к конечному набору: из всех опорных чисел
выбирался только конечный набор, который и подвергался затем процедуре
согласования.

Настоящая работа призвана восполнить этот недостаток: далее будет приведено
доказательство эквивалентности сильной и слабой согласованности теневых
контуров, которое служит обоснованием для применимости метода к рассматриваемой
практической задаче.

Следующим вопросом, который здесь будет рассмотрен, является метод быстрого
согласования опорных чисел, с помощью которого можно задачу в метрике
$L_{\infty}$ за линейное время, без использования линейного программирования.

\textbf{2. Основные понятия.}
Главным понятием, которое используется в данное статье, является понятие
\textit{опорной функции выпуклого тела}:

\begin{SmartDefinition}
 \label{def:support-function}
 \textbf{Опорной функцией} выпуклого тела $K \subset \mathbb{R}^{n}$
 называется
 $h_{K}: \mathbb{R}^{n} \to \mathbb{R}_{+}$:

 \begin{equation*}h_{K}(u) = \sup \limits_{x \in K}(x, u).\end{equation*}
\end{SmartDefinition}

В ряде практических задач, возникающих в компьютерной томографии
\cite{PrinceWillsky}, магнитно-резонансной визуализации
\cite{GregorRannou2001, GregorRannou2002} и обработке данных лазерного радара
\cite{LeleKulkarniWillsky}, встречаются такие, которые можно интерпретировать
как восстановление геометрической структуры выпуклого тела по измерениям его
опорной функции. А именно, речь идёт о следующей задаче: пусть имеется
некоторое реальное физическое тело $K$ и производятся эмпирические измерения
его опорной функции на конечном наборе единичных векторов
$u_{1}, u_{2}, \ldots, u_{m}$:

\begin{equation*}
 h_{i} = h_{K}(u_{i}) + \varepsilon_{i}, \;\; i = 1, 2, \ldots, m,
\end{equation*}

где $\varepsilon_{i}$ --- погрешность измерения величины $h_{i}$. Общепринятой
является следующая терминология: $h_{1}, h_{2}, \ldots, h_{m}$ суть
\textit{опорные числа} тела $K$ по \textit{опорным направлениям}
$u_{1}, u_{2}, \ldots, u_{m}$.

\begin{SmartDefinition}
 \label{def:consistency}
 Опорные числа $h_{1}, h_{2}, \ldots, h_{m}$ по направлениям \\
 $u_{1}, u_{2}, \ldots, u_{m}$ называются \textbf{согласованными}, если
 существует такое выпуклое тело  $K^{*} \subset \mathbb{R}^{n}$, что
 $h_{K^{*}}(u_{i}) = h_{i}$.
\end{SmartDefinition}

Таким образом, поскольку не всякий набор действительных чисел является
согласованным набором опорных чисел некоторого выпуклого тела, естественным
образом возникает задача \textit{согласования опорных чисел}: по набору опорных
чисел $h^{0}_{1}, h^{0}_{2}, \ldots, h^{0}_{m}$ и опорным направлениям
$u_{1}, u_{2}, \ldots, u_{m}$ строится такой согласованный набор опорных чисел
$h^{*}_{1}, h^{*}_{2}, \ldots, h^{*}_{m}$, который реализует минимум функционала

\begin{equation*}
 I = || h - h^{0} ||_{X},
\end{equation*}

где $X$ -- одна из метрик  в $\mathbb{R}^{m}$, например,
$L_{2}, L_{1}, L_{\infty}$.

В работе \cite{GardnerKiderlen} задача согласования опорных чисел сводится к
следующей задаче математического программирования:

\begin{equation}
\label{equation:gardner-kiderlen}
 ||\{(x_{i}, u_{i})\}_{i = 1}^{m} - h^{0}||_{X} \to \inf \;\;\; s. t. \;\;
 (x_{i}, u_{i}) \geq (x_{j}, u_{i}), 1 \leq i \neq j \leq m,
\end{equation}

где $x_{i} \in \mathbb{R}^{3}$ --- точки касания тела с опорной плоскостью,
имеющей нормаль $u_{i}$.


\textbf{3. Согласованность теневых контуров.}
Теневой контур $C$ выпуклого многогранника $K$ представляет из себя
многоугольник с конечным числом сторон, расположенный в некоторой фиксированной
плоскости $\pi$. Без ограничения общности можно считать, что плоскость контура
всегда содержит начало координат $O$. В таком случае, если измерения тела
производились абсолютно точно, и $C$ не содержит измерительной погрешности, то
можно утверждать что

\begin{equation*}
 h_{K}(u) = h_{C}(u) \;\; \forall u \in \pi,
\end{equation*}

то есть что опорные функции тела $K$ и его теневого контура $C$ совпадают на
большом круге сферы, образованном пересечением единичной сферы $||u||_{2} = 1$
и плоскости $\pi$. Таким образом, теневой контур $C$ можно интерпретировать как
непрерывное семейство измерений опорной функции тела $K$ на большом круге сферы.
Следовательно, имеет смысл говорить о \textit{согласованности теневых контуров}:

\begin{SmartDefinition}
 Пусть $C_{1}, C_{2}, \ldots, C_{n}$ -- теневые контуры по направлениям
 $\nu_{1}, \nu_{2}, \ldots, \nu_{n}$. Если существует такое выпуклое тело $K$,
 теневые контуры которого по данным направлениям в точности совпадают с
 $C_{1}, C_{2}, \ldots, C_{n}$, то данные теневые контуры называются
 \textit{согласованными}.
\end{SmartDefinition}

Аналогично конечным наборам опорных чисел, можно рассматривать задачи
согласования теневых контуров: с той лишь разницей, что целесообразным видится
ограничить семейство всех непрерывных опорных функций на больших сферах с
нормалями $\nu_{1}, \nu_{2}, \ldots, \nu_{n}$ более узким классом. Мы будем
рассматривать только такие теневые контуры, которые являются многоугольниками,
нормали сторон которых являются заранее известными и фискированы. Тем самым
априори будем предполагать, что исходное тело является многогранником с конечным
числом вершин, рёбер и граней.

Следующая теорема позволяет сводить задачу согласования теневых контуров к
известной задаче согласования конечных наборов опорных чисел:

\begin{SmartTheorem}
 Пусть $C_{1}, C_{2}, \ldots, C_{n}$ -- теневые контуры,
 $u^{i}_{j}$ -- нормаль к $j$-й стороне контура $C_{i}$, а $h^{i}_{j}$ --
 опорное число контура $C_{i}$ по направлению $u^{i}_{j}$.
 Тогда если опорные числа $\{h^{i}_{j}\}_{i, j}$ по направлениям
 $\{u^{i}_{j}\}_{i, j}$ согласованы, то согласованы и теневые контуры
 $C_{1}, C_{2}, \ldots, C_{n}$.
\end{SmartTheorem}

\textbf{Доказательство.}

Поскольку опорные числа $\{h^{i}_{j}\}_{i, j}$ по направлениям
$\{u^{i}_{j}\}_{i, j}$ согласованы, то пересечение
всех полупространств $(x, u^{i}_{j}) \leq h^{i}_{j}$ является тем самым телом
$K$, для которого $h_{K}(u^{i}_{j}) = h^{i}_{j}$.
Для всех $i, j$ обозначим $x^{i}_{j}$ -- точку касания тела $K$ с плоскостью
$(x, u^{i}_{j}) = h^{i}_{j}$.

Тогда для выполнены условия

\begin{equation*}
 (x^{i}_{j}, u^{i}_{j}) \geq (x^{i'}_{j'}, u^{i}_{j}) \;\;
 \forall i = 1, \ldots, n, \;
 j = 1, \ldots, m_{i}, \;
 i' = 1, \ldots, n, \;
 j' = 1, \ldots, m_{i'}
\end{equation*}

Очевидно, что в качестве точки касания по направлению $u^{i}_{j}$ можно
выбирать любую точку, лежащую на стороне контура $C_{i}$, имеющей нормаль
$u^{i}_{j}$. Для однозначности будем считать, что точки касания выбраны таким
образом, что $x^{i}_{j}$ является первой вершиной стороны контура, которая
встречается при его обходе по часовой стрелке.

Пусть $u$ -- некоторое единичное направление, лежащее в плоскости контура
$C_{i}$, а $x$ -- точка касания тела $K$ с плоскостью $(x, u) = h_{K}(u)$.
Очевидно, что $x$ является одной из точек $x^{i}_{1}, \ldots, x^{i}_{m_{i}}$.
Если вектор $u$ расположен между векторами $u^{i}_{j - 1}$ и $u^{i}_{j}$ в
плоскости контура $C_{i}$, то можно утверждать что $x = x^{i}_{j}$. Поскольку
$x^{i}_{j}$ лежит одновременно на $(j - 1)$-й и $j$-й сторонах контура, то
$(x^{i}_{j}, u^{i}_{j - 1}) = (x^{i}_{j - 1}, u^{i}_{j - 1})$.

Рассмотрим теперь некоторое другое единичное направление $u'$, лежащее в
плоскости другого контура $C_{k}$, и соответствующую ему точку касания $x'$.

Представим направление $u$ в виде линейной комбинации нормалей сторон контура
$C_{i}$, между которыми оно лежит:

\begin{equation*}
 u = \alpha u^{i}_{j - 1} + \beta u^{i}_{j},
\end{equation*}

причём $\alpha > 0, \beta > 0$, поскольку направление $u$ лежит между
$u^{i}_{j - 1}$ и $u^{i}_{j}$. Следовательно, справедлива следующая цепочка
неравенств:

\begin{align*}
 (x', u) &= (x, \alpha u^{i}_{j - 1} + \beta u^{i}_{j}) = \\
 = \alpha (x, u^{i}_{j - 1}) + \beta (x, u^{i}_{j}) &\leq
 \alpha (x^{i}_{j - 1}, u^{i}_{j - 1}) + \beta (x^{i}_{j}, u^{i}_{j}) = \\
 = \alpha (x^{i}_{j}, u^{i}_{j - 1}) + \beta (x^{i}_{j}, u^{i}_{j}) &=
 (x^{i}_{j}, \alpha u^{i}_{j - 1} + \beta u^{i}_{j}) = (x^{i}_{j}, u) = (x, u)
\end{align*}

Таким образом, доказано, что для любых направлений $u, u'$, лежащих в каких-либо
плоскостях (возможно, разных) теневых контуров $C_{1}, C_{2}, \ldots, C_{n}$,
справедливо неравенство:

\begin{equation*}
 (x', u) \leq (x, u),
\end{equation*}

где $x, x'$ -- точки касания тела $K$ по направлениям $u, u'$. Данное условие
в точности является критерием согласованности всякого набора опорных чисел, в
том числе непрерывного, что и требовалось доказать.

$\square$

Приведённая выше теорема позволяет обосновать применение уже существующих
разработанных методов согласования опорных чисел к задаче согласования теневых
контуров. Далее будет предложен новый алгоритм согласования.


\textbf{4. Быстрый алгоритм согласования.}

Основной конструкцией работы \cite{palachev} является переформулировка исходной
задачи согласования в двойственной постановке, т. е. путём отображения всех
исходных данных задачи и отношений между ними  под действием преобразования
двойственности. На этой конструкции был основан следующий результат:

\begin{SmartTheorem}\label{theorem:exhaustive-conditions}
Рассмотрим следующую систему ограничений:

\begin{equation}
\label{equation:all-constraints}
 (x_{i}, u_{i}) \geq (x_{i}, u_{j}), \;\;\;
 |(x_{i}, u_{i}) - h_{i}| \leq \varepsilon.
\end{equation}

Пусть система ограничений (\ref{equation:all-constraints}) разрешима
для некоторого $\varepsilon^{0}$. Обозначим многогранник

\begin{equation}
\label{equation:upper-bounds-polyhedron}
 K = \bigcap \limits_{i = 1}^{m}\{(x, u_{i}) \leq h_{i} + \varepsilon^{0}\}.
\end{equation}

Пусть $\delta$ - полярное преобразование двойственности.
Тогда если отрезок, соединяющий точки
$\delta(\{(x, u_{i}) = h_{i} - \varepsilon^{0}\})$ и
$\delta(\{(x, u_{j}) = h_{j} - \varepsilon^{0}\})$, пересекает
$\delta(K)$, то условия $(x_{i}, u_{i}) \geq (x_{j}, u_{i})$ и
$(x_{j}, u_{j}) \geq (x_{i}, u_{j})$ избыточны в системе ограничений
(\ref{equation:all-constraints}).
\end{SmartTheorem}

Рассмотрим задачу согласования опорных чисел в метрике $L_{\infty}$:

\begin{equation}
\label{equation:infinity-problem}
 \varepsilon \to \inf \;\;\;
 s. t. \;\; (x_{i}, u_{i}) \geq (x_{i}, u_{j}), \;\;\;
 |(x_{i}, u_{i}) - h_{i}| \leq \varepsilon
\end{equation}

По сути, данная задача линейного программирования состоит в нахождении такого
минимального $\varepsilon$, при котором система ограничений
(\ref{equation:all-constraints}) является разрешимой, т. е. когда существует
хотя бы один набор точек $(x_{1}, \ldots, x_{m})$, удовлетворяющий этой системе.
К данной задаче можно подойти с несколько иной стороны, если для каждого
фиксированного $\varepsilon$ выяснять, является ли система
(\ref{equation:all-constraints}) разрешимой.

\begin{SmartTheorem}
 Система ограничений (\ref{equation:all-constraints}) разрешима тогда и только
 тогда, когда ни одна из точек
 $\delta(\{(x, u_{i}) = h_{i} - \varepsilon\})$ не лежит во внутренности
 многогранника $\delta(K)$, где $K$ -- многогранник, определяемый соотношением
 (\ref{equation:upper-bounds-polyhedron}).
\end{SmartTheorem}

\textbf{Доказательство.}

Точка $\delta(\{(x, u_{i}) = h_{i} - \varepsilon\})$ не лежит во внутренности
многогранника $\delta(K)$ тогда и только тогда, когда плоскость
$\{(x, u_{i}) = h_{i} - \varepsilon\}$ пересекает многогранник $K$ (по
одному из свойств преобразования двойственности). Это в свою очередь
эквивалентно тому, что существует такая точка $x = x_{i}$, которая удовлетворяет
следующим соотношениям:

\begin{align*}
 (x_{i}, u_{i}) & \geq h_{i} - \varepsilon \\
 (x_{i}, u_{j}) & \leq h_{j} + \varepsilon, \; j = 1, \ldots, m.
\end{align*}

Без ограничения общности можно выбрать $x_{i}$ таким образом, что

\begin{equation*}
(x_{i}, u_{i}) = h_{i} + \varepsilon.
\end{equation*}

В таком случае, повторяя
приведённые выше рассуждения для всех $i = 1, \ldots, m$, мы получим набор
таких точек $(x_{1}, \ldots, x_{m})$, который будет удовлетворять системе
ограничений (\ref{equation:all-constraints}).

$\square$

Полученный результат позволяет организовать процесс нахождения решения задачи
линейного программирования \ref{equation:infinity-problem} с помощью
следующего алгоритма:

\begin{enumerate}
 \item По исходным опорным данным $u_{i}, h_{i}, i = 1, \ldots, m$
 построить тело $K^{0}$ как пересечение полупространств:
 \begin{equation*}
 \label{equation:naive-body}
  K^{0} = \bigcap \limits_{i = 1}^{m} \{(u_{i}, x) \leq h_{i}\}
 \end{equation*}
 \item Пусть $A_{1}, \ldots A_{s}$ --- вершины тела $K^{0}$. Вычислить опорные
 числа тела $K^{0}$ по направлениям $u_{i}, i = 1, \ldots, m$ следующим образом:
 \begin{equation*}
  h^{0}_{i} = \max \limits_{j = 1, \ldots, s} (u_{i}, A_{j})
 \end{equation*}
 \item Вычислить $L_{\infty}$-расстояние от полученных опорных чисел до
 исходных:
 \begin{equation*}
  \varepsilon^{0} = \max \limits_{i = 1, \ldots, m} |h_{i} - h^{0}_{i}|
 \end{equation*}
 Положить $\varepsilon_{-} = 0, \;\; \varepsilon_{+} = \varepsilon_{0}$
 \item Если $|\varepsilon_{+} - \varepsilon_{-}| < \Delta \varepsilon$, то
 остановить алгоритм, в противном случае положить
 $\varepsilon = \frac{\varepsilon_{-} + \varepsilon_{+}}{2}$ и перейти к
 следующему шагу.
 \item Вычислить точки
 $\delta(\{(x, u_{i}) = h_{i} + \varepsilon\}), \; i = 1, \ldots, m$ и построить
 их выпуклую оболочку $D$.
 \item Вычислить точки
 $\delta(\{(x, u_{i}) = h_{i} - \varepsilon\}), \; i = 1, \ldots, m$ и
 проверить, лежит ли хотя бы одна из них во внутренности $D$. Если хотя бы одна
 лежит, то изменить величину $\varepsilon_{-} = \varepsilon$, иначе изменить
 величину $\varepsilon_{+} = \varepsilon$. Перейти к шагу 4.
\end{enumerate}

В пункте 4 величина $\Delta \varepsilon$ является фиксирована, т. е. является
параметром алгоритма. определяющим точность вычисления минимального допустимого
$\varepsilon$. Количество итераций, необходимое для завершения алгоритма,
очевидно, равно $\log_{2} \frac{\varepsilon_{0}}{\Delta \varepsilon}$. На каждом
шаге алгоритма требуется $2m$ раз вычислять двойственные образы плоскостей (за
константное число операций), $1$ раз строить выпуклую оболочку $m$ точек
(что требует $O(m \log m)$ времени) и $m$ раз проверять принадлежность точки
построенному многограннику (что требует $O(\log m)$ времени). Следовательно,
трудоёмкость работы алгоритма составляет

\begin{equation*}
 \log_{2} \frac{\varepsilon_{0}}{\Delta \varepsilon} \cdot O(m \log m).
\end{equation*}

Поскольку в реальных задачах величины $\varepsilon_{0}$ и
$\Delta \varepsilon$ различаются не больше чем на несколько порядков, то
можно считать, что представленный алгоритм имеет линейно-логарифмическую
сложность.

\textbf{5. Реализация и тестирование алгоритма}.
Реализация была произведена на основе библиотеки алгоритмов вычислительной
геометрии CGAL \cite{cgal}. Пересечение полупространств в пункте 1 и выпуклая
оболочка в пункте 5 вычислялись на основе алгоритма построения трехмерной
триангуляции Делоне. Выбор в пользу данного алгоритма был сделан, поскольку он
позволяет производить быструю многократную проверку принадлежности точки к
выпуклой оболочке.

Тестирование было произведено на опорных числах, полученных из реальных
теневых контуров. В метрике $L_{\infty}$ алгоритм работает приблизительно в 300
раз быстрее, чем существующий алгоритм решения задачи линейного программирования
на основе метода внутренней точки. К примеру, решение задачи на 200 теневых
контурах, из которых были получены 2912 опорные точек, было получено за 0,98
секунды, в то время как предыдущий алгоритм позволял получить решение лишь за
299,59 секунд.

Поскольку на практике наиболее значимым является решение задачи именно в
$L_{2}$-метрике, существующие методы решение задачи остаются востребованными.
Впрочем, решение в $L_{\infty}$ метрике можно использовать в качестве начального
приближения в итерационном процессе решения задачи квадратичного
программирования.

Представляется также перспективной попытка разработать на основе представленного
выше метода новый метод решения задачи в метриках $L_{1}$ и $L_{2}$ путём
последовательного нахождения тех опорных чисел, для которых нарушение
разрешимости системы ограничений (\ref{equation:all-constraints}) при
непрерывном уменьшении $\varepsilon$ наступает раньше всех остальных. Такие
точки, как кажется, является сообразным представить в виде линейной
комбинации соседних опорных чисел.


\begin{thebibliography}{50}

\bibitem{palachev}
\emph{Палачев~И.А.}
Метод исключения избыточных ограничений в задаче восстановления тела по
измерениям его опорной функции //
Вычислительные методы и программирование.
2015.
\textbf{16}.
348--359.

\bibitem{PrinceWillsky}
\emph{Prince~J.L., Willsky~A.S.}
Reconstructing convex sets from support line measurements //
IEEE Transactions on Pattern Analysis and Machine Intelligence.
1990.
\textbf{12},
N 4.
377--389.

\bibitem{LeleKulkarniWillsky}
\emph{Lele~A.S., Kulkarni~S.R., Willsky~A.S.}
Convex-polygon estimation from support-line measurements and applications to
target reconstruction from laser-radar data //
Journal of the Optical Society of America A.
1992
\textbf{9},
N 10.
1693--1714.

\bibitem{GregorRannou2001}
\emph{Gregor~J., Rannou~F.R.}
Least-squares framework for projection {MRI} reconstruction //
Proceedings of SPIE.
2001.
\textbf{4322}.
888-898.

\bibitem{GregorRannou2002}
\emph{Gregor~J., Rannou~F.R.}
Three-dimensional support function estimation and application for projection
magnetic resonance imaging //
International Journal of Imaging Systems and Technology.
2002.
\textbf{12},
N 1.
43--50.

\bibitem{GardnerKiderlen}
\emph{Gardner~R.J., Kiderlen~M.}
A new algorithm for 3D reconstruction from support functions //
IEEE Transactions on Pattern Analysis and Machine Intelligence.
2009.
\textbf{31},
N 3.
556--562.

\bibitem{cgal}
\textsc{Cgal}, {C}omputational {G}eometry {A}lgorithms {L}ibrary.
\url{http://www.cgal.org}

\end{thebibliography}

\end{document}