\documentclass[a4paper,12pt, titlepage]{article} 
\usepackage{geometry}           % пакет для задания полей страницы командой \geometry
\geometry{left=3cm,right=1.5cm,top=2cm,bottom=2cm}
\usepackage[utf8]{inputenc}
\usepackage[english,russian]{babel}
\usepackage{amsmath}
%\usepackage{amsthm}
%\usepackage{cmap}
\usepackage{indentfirst}
\usepackage{a4wide,amssymb}
%\usepackage[pdftex]{graphicx}
%\usepackage[pdftex]{graphics}
%\usepackage{wrapfig}
%\linespread{1.3}                % полтора интервала. Если 1.6, то два интервала
\pagestyle{plain}               % номерует страницы

\usepackage{graphicx}
\renewcommand{\topfraction}{1}
\renewcommand{\textfraction}{0}


%opening
\title{Восстановление многогранника по набору его теневых контуров \\ Обзор материалов и план работы}
\author{Палачев Илья}

\begin{document}

\maketitle

\tableofcontents

\section{Постановка задачи}


Имеется некоторый физический камень, геометрическая форма которого
представляет собой многогранник. Имеется установка, которая позволяет получать
информацию о его форме следующим образом:
 
\begin{enumerate}
  \item Камень жестко закрепляется одной своей гранью на горизонтальной
  подставке.
  \item Производится фотографирование камня вдоль некоторого горизонтального
  фиксированного направления $\nu_{0}$.
  \item Результатом фотографирования является монохромное изображение без
  каких-либо внутренних ребер, иными словами, "тень" камня.
  \item Затем подставка вместе с закрепленным на ней камнем поворачивается на
  фиксированный угол $\alpha_{0}$ и получается новая фотография камня.
\end{enumerate}
 
Этот процесс повторяется $K = \frac{2 \pi}{\alpha_{0}}$ раз, пока не получатся
фотографии камня по всем направлениям $\alpha_{k} = k \alpha_{0}$, где
$k = 1, 2, \ldots, K$. Все эти фотографии подвергаются обработке, в результате
которых получаются так называемые \textbf{теневые контуры} -- многоугольники,
лежащие в плоскостях, ортогональных горизонтальным векторам, образующим углы
$\alpha_{k} = k \alpha_{0}$ с осью $Ox$. 




\section{Исторический обзор смежных вопросов}

\section{Предлагаемый подход}

\end{document}
