\documentclass[a4paper,12pt, titlepage]{article} 
\usepackage{geometry}           % пакет для задания полей страницы командой \geometry
\geometry{left=3cm,right=1.5cm,top=2cm,bottom=2cm}
\usepackage[utf8]{inputenc}
\usepackage[english,russian]{babel}
\usepackage{amsmath}
%\usepackage{amsthm}
%\usepackage{cmap}
\usepackage{indentfirst}
\usepackage{a4wide,amssymb}
%\usepackage[pdftex]{graphicx}
%\usepackage[pdftex]{graphics}
%\usepackage{wrapfig}
%\linespread{1.3}                % полтора интервала. Если 1.6, то два интервала
\pagestyle{plain}               % номерует страницы

\usepackage{graphicx}
\renewcommand{\topfraction}{1}
\renewcommand{\textfraction}{0}


%opening
\title{Восстановление многогранника по набору его теневых контуров \\ Обзор материалов и план работы}
\author{Палачев Илья}

\begin{document}

\maketitle

\tableofcontents

\section{Постановка задачи}


Имеется некоторый физический камень, геометрическая форма которого
представляет собой многогранник. Имеется установка, которая позволяет получать
информацию о его форме следующим образом:
 
\begin{enumerate}
  \item Камень жестко закрепляется одной своей гранью на горизонтальной
  подставке.
  \item Производится фотографирование камня вдоль некоторого фиксированного
  горизонтального направления $\nu_{0}$.
  \item Результатом фотографирования является монохромное изображение без
  каких-либо внутренних ребер, иными словами, "тень" камня.
  \item Затем подставка вместе с закрепленным на ней камнем поворачивается на
  фиксированный угол $\alpha_{0}$ в горизонтальной плоскости и процесс
  повторяется начиная с пункта 2.
\end{enumerate}
 
Этот процесс повторяется $K = \frac{2 \pi}{\alpha_{0}}$ раз, пока не получатся
фотографии камня по всем направлениям $\alpha_{k} = k \alpha_{0}$, где
$k = 1, 2, \ldots, K$. Все эти фотографии подвергаются обработке, в результате
которых получаются так называемые \textbf{теневые контуры} -- многоугольники,
лежащие в плоскостях, ортогональных горизонтальным векторам, образующим углы
$\alpha_{k} = k \alpha_{0}$ с осью $Ox$. 

Требуется построить трехмерный многогранник, форма которого наилучшим образом
соответствует исходному камню. При этом под качеством построенной модели
подразумевается следующее:

\begin{enumerate}
 \item Тени модельного многогранника наилучшим образом приближают тени,
 полученные из измерений.
 \item В многограннике характерные вершины и грани не разделены на несколько
 вершин и граней.
\end{enumerate}

\section{Исторический обзор смежных вопросов}

Далее приводится обзор некоторых статей, которые были написаны в целях решения
задач, схожих с рассматриваемой. Как оказалось, особенно большое сходство было
найдено с задачей оценки опорной функции в геометрической томографии. В целях
последовательности изложения сначала приводится понятие опорной функции, а
затем задачи и алгоритмы их решения.

\subsection{Понятие опорной функции выпуклого тела}

Для простоты будем рассматривать выпуклые тела в трехмерном пространстве
$\mathbb{R}^{3}$, содержащие в своей внутренности начало координат $O$.
Для некоторых понятий будем давать определения и формулировки для случая
произвольной конечной размерности.

Как известно, выпуклое тело можно однозначно представить как пересечение 
полупространств всех его касательных плоскостей 
$K = \bigcap \limits_{x \in K} R_{x}$, где $R_{x}$ -- то из двух
полупространств, на которые плоскость $\pi_{x}$, касательная к телу $K$ в
точке $x$, делит $\mathbb{R}^{3}$, которое содержит в себе целиком все тело
$K$. Всякому выпуклому телу можно поставить в соответствие набор касательных 
плоскостей $\pi_{x}$, по которым его можно построить. Обратное неверно: не
всякому произвольному набору плоскостей можно поставить в соответствие тело,
касающееся всех этих плоскостей.

Всякую касательную плоскость $\pi_{x}$ можно однозначно охарактеризовать
единичным вектором нормали $u_{x}$ и расстоянием $h_{x}$ от начала координат 
$O$ до плоскости. Поскольку две разные касательные плоскости не могут иметь
одинаковые векторы нормалей, то можно рассматривать множество всех касательных
плоскостей выпуклого тела как функцию, определенную на всех единичных векторах
$u \in S_{2}$:

$$h_{K}: S^{2} \to \mathbb{R}_{+}$$

Более общее понятие включающее в себя выше указанное было введено в 1903 году
Минковским.

\begin{flushleft}
\textbf{Определение}. Будем называть \textbf{опорной функцией} выпуклого тела
$K \subset \mathbb{R}^{n}$ следующую функцию 
$h_{K}: \mathbb{R}^{n} \to \mathbb{R}_{+}$:

$$h_{K}(u) = \max \limits_{x \in K}(x, u)$$
\end{flushleft}

Если взять некоторую точку $u_{0}$ на единичной сфере $S^{n - 1}$, и вычислить
в ней значение опорной функции $h_{K}(u_{0})$, то по этим данным можно
построить касательную плоскость к выпуклому телу в некоторой (неизвестной!)
точке $x \in K$. Такую плоскость (в контексте, когда нет информации о положении
точки касания) принято называть опорной плоскостью.

\begin{flushleft}
 \textbf{Определение}. \textbf{Опорной плоскотью} выпуклого тела $K$ по
 направлению $u \in S^{n - 1}$ называется плоскость с нормалью $u$, расстояние
 от которой до начала координат равно $h_{K}(u)$
\end{flushleft}

Очевидно, что опорная функция выпуклого тела обладает следующим свойством:

$$h_{K}(\lambda u) = \lambda h_{K}(u)$$

Следовательно, для практики достаточно иметь дело только с ограничением опорной
функции на единичную сферу. В статье \cite{journals/jmiv/KarlKVW96} вводится
понятие \textbf{приведенной опорной функции}:

$$H_{K} (u) = h_{K} (\frac{u}{||u||})$$

которая в действительности педставляет собой расстояние от начала координат
$\mathbb{O}$ до опорной гиперплоскости по направлению $u$.

Более подробно свойства опрной функции рассматриваются в статье
\cite{journals/cviu/GhoshK98}.

%(TODO: прочесть и законспектировать)

\subsection{Восстановление выпуклого многоугольника по измерениям его опорной
функции (по статье Prince, Willsky)}

В работе \cite[Prince - Willsky (1990)]{journals/pami/PrinceW90}
рассматриваются алгоритмы для восстановления двумерных выпуклых тел по
измерениям их опорных функций. Изначально изучение данной проблемы было
мотивировано задачей из компьютерной томографии. А именно, в томографии
делаются измерения интегралов плотности поглощения излучения объектом вдоль
различных фиксированных прямых. Допустим, что известны интегралы плотности
поглощения по пучку прямых $L(t, \theta)$, где угол $\theta$ фиксирован. Тогда
по этой информации можно определить положение двух опорных прямых к данному
объекту.

По имеющимся измерениям опорной функции можно потроить грубое приближение
расмматриваемого тела -- путем обыкновенного пересечения полуплоскостей,
соответствующих опорным прямым. Однако на практике измерения подвержены
ошибкам и известны лишь с некоторой точностью. Так, может оказаться, что
после пересечения построенное тело будет касаться не всех заданных прямых.
Это приводит к тому, что всего одно грубое измеренние может заблокировать
воздействие других (более точных) измерений на результат.




\section{Предлагаемый подход}

\newpage
\bibliographystyle{plain}
\bibliography{references}

\end{document}
