
% Copyright (c) 2009-20134 Ilya Palachev <iliyapalachev@gmail.com>
% This document contains the report about reconstruction of convex sets.

% Declare the class of document: size of paper, size of font, and etc.
% Type of the document is "article".
\documentclass[a4paper, 12pt, titlepage]{article} 

% Package that enables setting the size of free spaces at the border of the 
% page with the command  \geometry :
\usepackage{geometry}
\geometry {
   left=3cm,
   right=1.5cm,
   top=2cm,
   bottom=2cm
}

\usepackage[utf8]{inputenc}

\usepackage[english,russian]{babel}

\usepackage{amsmath}

%\usepackage{cmap}

\usepackage{indentfirst}

\usepackage{a4wide,amssymb}

%\usepackage[pdftex]{graphicx}

%\usepackage{wrapfig}

%\linespread{1.3}               % полтора интервала. Если 1.6, то два интервала
\pagestyle{plain}               % номерует страницы

\usepackage{graphicx}
\renewcommand{\topfraction}{1}
\renewcommand{\textfraction}{0}

% Package that enables usage of theorems and definition designed in a 
% standard way.
% http://en.wikibooks.org/wiki/LaTeX/Theorems
\usepackage{amsthm}

% Create environment for smart definitions
\theoremstyle{definition}
\newtheorem{SmartDefinition}{Определение}

% Create environment for smart theorems
\theoremstyle{plain}
\newtheorem{SmartTheorem}{Теорема}

% Create environment for smart lemmas
\theoremstyle{plain}
\newtheorem{SmartLemma}{Лемма}

% The following code enables back references:
\usepackage{color} 
\definecolor{darkgreen}{rgb}{0,.5,0} 
\usepackage[unicode,colorlinks,filecolor=blue,citecolor=darkgreen,pagebackref]
{hyperref}

% The package that provides symbols like \Square:
\usepackage{wasysym}

%opening
\title{Отчет по методам оптимизации}
\author{Палачев Илья}

\begin{document}

\maketitle

\tableofcontents

\newpage
\section{Линейное программирование}
Одним из самых распространенных классов задач оптимизации являются задачи
\textbf{линейного программирования}. Методы решения этих задач имеют широкое
приложение в разнообразных прикладных областях науки. Самым распространенным
методом решения задач линейного программирования является так называемый
\textbf{симплекс-метод}, который будет далее описан в этом разделе.

\textit{Общая задача линейного программирования} формулируется следующим
образом.

\begin{equation}
\label{eq:gen-lp}
 J(u) = c_{1} u^{1} + \ldots + c_{n} u^{n} \to min
\end{equation}

при условиях

\begin{equation}
\label{eq:gen-lp-cond1}
 u^{k} \geq 0, \;\;\;\;\; k \in I
\end{equation}

\begin{equation}
\label{eq:gen-lp-cond2}
 \left\{
  \begin{aligned}
   a_{11} u^{1} + \ldots + a_{1n} u^{n} \leq 0 \\
   \ldots \\
   a_{m1} u^{1} + \ldots + a_{mn} u^{n} \leq 0 \\
  \end{aligned}
 \right.
\end{equation}

\begin{equation}
\label{eq:gen-lp-cond3}
 \left\{
  \begin{aligned}
   a_{m + 1, 1} u^{1} + \ldots + a_{m + 1, n} u^{n} = 0 \\
   \ldots \\
   a_{s1} u^{1} + \ldots + a_{sn} u^{n} = 0 \\
  \end{aligned}
 \right.
\end{equation}

где $I \subset \{1, 2, \ldots, n\}$ - множество индексов переменных, которые
требуется поддерживать положительными.

В векторном виде задачу (\ref{eq:gen-lp}), (\ref{eq:gen-lp-cond1}),
(\ref{eq:gen-lp-cond2}), (\ref{eq:gen-lp-cond3}) можно записать следующим
образом:

\begin{equation}
 \begin{aligned}
  J(u) = \langle c, u \rangle \to min \\
  u \in U = \{u \in \mathbb{R}^{n} \;\; | \;\; u^{k} \geq 0, k \in I,
  A u \leq 0, \overline{A} u \overline{b}\}
 \end{aligned}
\end{equation}

В качестве особого подкласса общей задачи линейного программирования
выделяют так называемую \textit{каноническую задачу}:

\begin{equation}
 \begin{aligned}
  J(u) = \langle c, u \rangle \to min \\
  u \in U = \{u \in \mathbb{R}^{n} \;\; | \;\; u \geq 0,
  \overline{A} u = \overline{b}\}
 \end{aligned}
\end{equation}

а также \textit{основную задачу} линейного программирования:

\begin{equation}
 \begin{aligned}
  J(u) = \langle c, u \rangle \to min \\
  u \in U = \{u \in \mathbb{R}^{n} \;\; | \;\; u \geq 0,
  \overline{A} u \leq \overline{b}\}
 \end{aligned}
\end{equation}

Можно показать, что каждую из трех задач можно свести к любой другой с помощью
введения дополнительных переменных. На практике как правило к этому стараются
не прибегать, поскольку рост числа переменных обходится дорого с вычислительной 
точки зрения. Далее мы будем рассматривать методы решения канонической задачи.

Из геометрических соображений видно, что множество ограничений $U$ является
многогранным, причем либо ограниченным, либо неограниченным. Если рассмотреть
поверхности уровня минимизируемого функционала, которые являются аффинными
подпространствами в $\mathbb{R}^{n}$, то пересекая их с множеством $U$, мы
сможем выявить поверхность самого низкого уровня. На ней и будет достигаться
минимум функционала. Это поверхность будет являться опорной к множеству
ограничений, поэтому в ее пересечении с $U$ будет содержаться хотя бы одна 
вершина $U$. В теории линейного программирования такие вершины принято
называть \textit{угловыми}.

\begin{SmartDefinition}
 Пусть $U$ -- выпуклое множество в $\mathbb{R}^{n}$. Тогда $v \in U$ называется
 \textbf{угловой точкой} множества $U$, если представление
 $v = \alpha v_{1} + (1 - \alpha) v_{2}$ при $v_{1}, v_{2} \in U$ и
 $0 < \alpha < 1$ возможно лишь при $v_{1} = v_{2}$. Иначе говоря $v$ -- угловая
 точка множества $U$, если она не является внутренней точкой никакого отрезка,
 принадлежащего множеству $U$.
\end{SmartDefinition}

В случае канонической задачи множество $U$ представляет из себя следующее:

\begin{equation}
\label{eq:canon-cond}
 U = \{u \in \mathbb{R}^{n} \;\; | \;\; u \geq 0, A u = b\}
\end{equation}

Критерий того, что точка является угловой точкой множества определяет следующая
теорема.

\begin{SmartTheorem}
 Пусть множество определяется условиями (\ref{eq:canon-cond}), $A \neq 0$,
 $r = rang A$ -- ранг матрицы $A$. Для того, чтобы точка
 $v = (v_{1}, v_{2}, \ldots, v_{n})$ была угловой точкой множества $U$,
 необходимо и достаточно, чтобы существовали набор из $r$ номеров
 $j_{1}, \ldots, j_{r}$ ($1 \leq j_{l} \leq n, l = 1, \ldots, r$) таких, что
 \begin{equation}
 \label{eq:vertex-def}
  A_{j_{1}} v^{j_{1}} + \ldots + A_{j_{r}} v^{j_{r}} = b;
  \;\; v^{j} = 0, j \neq j_{l}, l = 1, \ldots, r
 \end{equation}
\end{SmartTheorem}

В последнем определении существенно, что в (\ref{eq:vertex-def}) берутся ровно
$r$ векторов $A_{j_{1}}, \ldots, A_{j_{r}}$ -- ровно такое число векторов,
каков ранг матрицы $A$.

\begin{SmartDefinition}
 Систему векторов $A_{j_{1}}, \ldots, A_{j_{r}}$, входящих в первое из равенств
 (\ref{eq:vertex-def}), называют \textit{базисом угловой точки} $v$, а
 соответствующие им величины $v^{j_{1}}, \ldots, v^{j_{r}}$ --
 \textit{базисными координатами} угловой точки $v$. Если все базисные 
 координаты угловой точки положительны, то такую угловую точку называют
 \textit{невырожденной}. В противном случае -- \textit{вырожденной}. При
 фиксированном базисе $A_{j_{1}}, \ldots, A_{j_{r}}$ переменные
 $u^{j_{1}}, \ldots, u^{j_{r}}$ называются \textit{базисными переменными}
 угловой точки, а остальные $u^{j}$ -- \textit{небазисными (свободными)
 переменными}.
\end{SmartDefinition}

Для того чтобы избежать трудоемкого полного перебора всех угловых точек
множества ограничений $U$, выбирают одну, от нее переходят к другой, у которой
значение функционала меньше и т. д. В этом и состоит симплекс-метод.

Далее будем считать, что $m = n$ и все уравнения ограничений, линейно зависимые
от других, выброшены.

Пусть нам известна угловая точка $v$. Пусть $\overline{u} = (u_{1},
u_{r})$ -- вектор, составленный из базисных переменных угловой точки
$v$. Перепишем систему ограничений в следующем виде:

\begin{equation}
 B \overline{u} + A_{r + 1} u^{r + 1} + \ldots + A_{n} u^{n} = b
\end{equation}

Домножим эту систему слева на $B^{-1}$, базисные переменные через
небазисные и получим следующее:

\begin{equation}
\label{eq:nonfree-via-free}
 \begin{aligned}
  u^{1} = v^{1} - \gamma_{1, r + 1} u^{r + 1} - \ldots - \gamma_{1 k} u^{k} -
  \ldots - \gamma_{1 n} u^{n} \\
  \ldots \\
  u^{i} = v^{i} - \gamma_{i, r + 1} u^{r + 1} - \ldots - \gamma_{i k} u^{k} -
  \ldots - \gamma_{i n} u^{n} \\
  \ldots \\
  u^{r} = v^{r} - \gamma_{r, r + 1} u^{r + 1} - \ldots - \gamma_{r k} u^{k} -
  \ldots - \gamma_{r n} u^{n} \\
 \end{aligned}
\end{equation}

Подставив эти выражения в формулу функционала, получим:

\begin{equation}
 J(u) = J(v) - \sum \limits_{i = r + 1}^{n} \Delta_{i} u^{i}
\end{equation}

где

\begin{equation}
 \Delta_{i} = \langle \overline{c}, B^{-1} A_{i} \rangle =
 \sum \limits_{s = 1}^{r} c_{s} \gamma_{s i} - c_{i}
\end{equation}

Запишем величины $\gamma_{s k}, v^{i}, \Delta_{i}$ в так называемую
симплекс-таблицу:

\begin{table}[hh]
\caption{Симплекс-таблица}
\label{simplex-table}
\begin{center}
\begin{tabular}{|p{2cm}|c|c|c|c|c|c|c|c|c|c|c|c|c|c|p{2cm}|}
\hline
Базисные \par переменные & $u^{1}$ & $\ldots$ & $u^{i}$ & $\ldots$ &
$u^{s}$ & $\ldots$ & $u^{r}$ & $u^{r + 1}$ & $\ldots$ & $u^{k}$ & $\ldots$ &
$u^{j}$ & $\ldots$ & $u^{n}$ & Свободные \par члены \\
\hline
$u^{1}$ & 1 & $\ldots$ & 0 & $\ldots$ & 0 & $\ldots$ & 0 & $\gamma_{1, r + 1}$ &
$\ldots$ & $\gamma_{1k}$ & $\ldots$ & $\gamma_{1j}$ & $\ldots$ & $\gamma_{1n}$
& $v^{1}$ \\
$\ldots$ & $\ldots$ & $\ldots$ & $\ldots$ & $\ldots$ & $\ldots$ & $\ldots$ &
$\ldots$ & $\ldots$ & $\ldots$ & $\ldots$ & $\ldots$ & $\ldots$ & $\ldots$ &
$\ldots$ & $\ldots$ \\
$u^{i}$ & 0 & $\ldots$ & 1 & $\ldots$ & 0 & $\ldots$ & 0 & $\gamma_{i, r + 1}$ &
$\ldots$ & $\gamma_{i k}$ & $\ldots$ & $\gamma_{i j}$ & $\ldots$ &
$\gamma_{i n}$ & $v^{i}$ \\
$\ldots$ & $\ldots$ & $\ldots$ & $\ldots$ & $\ldots$ & $\ldots$ & $\ldots$ &
$\ldots$ & $\ldots$ & $\ldots$ & $\ldots$ & $\ldots$ & $\ldots$ & $\ldots$ &
$\ldots$ & $\ldots$ \\
$u^{s}$ & 0 & $\ldots$ & 0 & $\ldots$ & 1 & $\ldots$ & 0 & $\gamma_{s, r + 1}$ &
$\ldots$ & $\gamma_{s k}$ & $\ldots$ & $\gamma_{s j}$ & $\ldots$ &
$\gamma_{s n}$ & $v^{s}$ \\
$\ldots$ & $\ldots$ & $\ldots$ & $\ldots$ & $\ldots$ & $\ldots$ & $\ldots$ &
$\ldots$ & $\ldots$ & $\ldots$ & $\ldots$ & $\ldots$ & $\ldots$ & $\ldots$ &
$\ldots$ & $\ldots$ \\
$u^{r}$ & 0 & $\ldots$ & 0 & $\ldots$ & 0 & $\ldots$ & 1 & $\gamma_{r, r + 1}$ &
$\ldots$ & $\gamma_{r k}$ & $\ldots$ & $\gamma_{r j}$ & $\ldots$ &
$\gamma_{r n}$ & $v^{r}$ \\
\hline
Функция & 0 & $\ldots$ & 0 & $\ldots$ & 0 & $\ldots$ & 0 & $\Delta_{r + 1}$ &
$\ldots$ & $\Delta_{k}$ & $\ldots$ & $\Delta_{j}$ & $\ldots$ & $\Delta_{n}$ &
$J(v)$ \\
\hline
\end{tabular}
\end{center}
\end{table}

Опишем коротко алгоритм линейного программирования. Допустим, что множество $U$
непусто и нам уже известна некоторая угловая точка $v$.

Будем менять по очереди небазисные переменные, тогда функционал будет меняться
по формуле

\begin{equation}
 J(u) = J(v) - \Delta_{k} u^{k}
\end{equation}

Будем искать такой номер $k$ ($r + 1 \leq k \leq n$) и такую величину $u^{k}$,
чтобы точка $w$, полученная по формулам (\ref{eq:nonfree-via-free}) для набора
свободных переменных $0, \ldots, 0, u_{k}, 0, \ldots, 0$, удовлетворяла
условиям связи $A w = b$. В зависимости от знаков величин $\gamma_{s k},
\Delta_{i}$ выделяют три возможных случая:

\textbf{Случай 1.} $\Delta_{i} \leq 0$ для всех $i = r + 1, \ldots, n$. Тогда 
можно показать, что $v$ -- точка минимума функционала.

\textbf{Случай 2.} Существует такой номер $k$ ($r + 1 \leq k \leq n$), что 
$\Delta_{k} > 0$, но все $\gamma_{i k} \leq 0$ для всех $i = 1, \ldots, r$.
Тогда можно показать, что функционал не является ограниченным снизу на $U$.

\textbf{Случай 3.} Существуют номера $k$ ($r + 1 \leq k \leq n$),
$i$ ($1 \leq i \leq r$) такие что $\Delta_{k} > 0$ и $\gamma_{i k} > 0$.
Для того чтобы обеспечить выполнение условий связи полагают
$u^{k} = v_{s} / \gamma_{s k}$, где индекс $s$ выбран следующим образом:

\begin{equation}
 \operatornamewithlimits{min}_{i \in I_{k}} v^{i} / \gamma_{i k} =
 v_{s} / \gamma_{s k}
\end{equation}

Соответствующую величину $\gamma_{s k}$ называют \textit{разрешающим элементом}
симплекс-таблицы.

Можно показать что полученная таким образом точка $w$ является угловой, 
реализует меньшее значение функционала, чем точка $v$, и что вектора
$A_{1}, \ldots, A_{s - 1}, A_{s + 1}, \ldots, A_{r}, A_{k}$ образуют ее базис.
После этого пересчитывают величины симплекс-таблицы по тем же принципам,
которые описаны выше, и вновь ищут следующую угловую точку.

Описанная выше процедура работает для случая, когда точка $v$ является
невырожденной. Если же она вырождена, то во время работы алгоритма может
произойти \textit{зацикливание}. Для того, чтобы его избежать, применяют
специальный метод, называемый \textbf{антициклином}. Он состоит в том, что
симплекс-таблица расширяется дополнительными столбцами единичной матрицы
порядка $r \times r$. На каждом шаге эти столбцы преобразуются по тем же
правилам, что и обычные. В случае вырожденной угловой точки эти столбцы
позволяют выбрать такой особый базис угловой точки $v$ (которых, вообще говоря,
должно быть несколько), что зацикливания можно избежать.

Указанные выше построения основывались на том, что некоторая угловая точка
множества $U$ уже известна. Для того, чтобы найти первую угловую точку также
применяют особый метод.

А именно, множество переменных $u = (u^{1}, \ldots, u^{n})$ расширяется
дополнительными переменными $w = (u^{n + 1}, \ldots, u^{n + m})$ и в
пространстве $\mathbb{R}^{n + m}$ относительно переменных
$z = (u, w)^{T}$ рассматривают следующую каноническую задачу линейного
программирования:

\begin{equation}
\begin{aligned}
 J_{1}(z) = u^{n + 1} + \ldots + u^{n + m} \to min \\
 z \in Z = \{ z \in \mathbb{R}^{n + m} | z \geq 0,
 C z \equiv A u + w = b \} \\
\end{aligned}
\end{equation}

Оказывается, что для нее точка $z_{0} = (0, b) \geq 0$ является угловой точкой
множества $Z$. Очевидно, что $J_{1}(z) \geq 0$ при всех $z \in Z$. Поэтому
алгоритм симлекс-метода обязательно сходится. Пусть он сойдется к некоторой
точке $z_{*}$. Тогда можно показать, что если $J_{1}(z_{*}) > 0$, то
$U = \varnothing$, а если $J_{1}(z_{*}) = v^{n + 1} + \ldots + v^{n + m} = 0$, 
так что $z = (v, 0)$, где $v$ -- угловая точка множества $U$.



\section{Рассматриваемые задачи}
\label{sec:problem}

В предыдущем отчете было указано, что задача восстановления многогранника по
набору его теневых контуров сводится в том или ином виде к задаче квадратичной
минимизации. В рассматриваемом случае функционал представлял из себя сумму
квадратов разностей опорных чисел и их экспериментальных измерений:

\begin{equation}
\label{eq:problem1}
 I(h) = I(h_{1}, h_{2}, \ldots, h_{n}) = ||h - h^{0}||_{2}^{2} =
 \sum \limits_{i = 1}^{n} (h_{i} - h_{i}^{0})^{2} \to min
\end{equation}

где числа $h_{i}^{0}$ есть экспериментальные измерения величин $h_{i}$.

При этом требовалось, чтобы вектор опорных чисел удовлетворял условию

\begin{equation}
\label{eq:problem1-cond}
 Q h \geq 0
\end{equation}

где матрица $Q$ представляла собой матрицу коэффициентов линейных неравенств
следующего типа:

\begin{equation}
\left|\begin{array}{cc}
  h_{1} & u_{1}^{T} \\
  h_{2} & u_{2}^{T} \\
  h_{3} & u_{3}^{T} \\
  h_{4} & u_{4}^{T} \\
\end{array}\right|
  \left|\begin{array}{cc}
  1 & u_{1}^{T} \\
  1 & u_{2}^{T} \\
  1 & u_{3}^{T} \\
  1 & u_{4}^{T} \\
\end{array}\right|
\geq 0
\end{equation}

где $u_{i}^{T}$ -- известные опорные направления, а $h_{i}$ -- неизвестные
опорные числа. Таким образом, матрица $Q$ представляет собой $m \times
n$-матрицу, причем в каждой ее строке содержится не более 4 ненулевых элементов,
и число $m$ ее строк не превышает $12 n$ (в соответствии с теоремой Эйлера о
числе вершин, ребер и граней графа на сфере).

Задачу минимизации (\ref{eq:problem1}) - (\ref{eq:problem1-cond}) можно
сформулировать несколько более удобным образом, как это описано, например, в
книге \cite{BertsekasTsitsiklis1989}: оптимальное решение задачи
(\ref{eq:problem1}) можно найти как

\begin{equation}
\label{eq:problem2}
 h^{*} = h^{0} - Q^{T} u^{*}
\end{equation}

где $u^{*}$ есть решение задачи минимизации функционала

\begin{equation}
\label{eq:problem2-cond}
 I(u_{1}, u_{2}, \ldots, u_{m}) = ||Q^{T} u - h^{0}||_{2}^{2} \to min
\end{equation}

при условиях неотрицательности переменных:

\begin{equation}
 u \geq 0
\end{equation}

Данный отчет посвящен поиску методов, которые позволили бы решить задачи
(\ref{eq:problem1}) - (\ref{eq:problem1-cond}) или (\ref{eq:problem2}) -
(\ref{eq:problem2-cond}) наиболее эффективно.

\section{Метод нормальных уравнений}

Если говорить о задачах наименьших квадратов вообще, то их можно выделить в
особый тип задач минимизации. Предположим, что матрица $A \in \mathbb{R}^{m
\times n}$ и $m > n$. В общем случае невозможно точно решить уравнение
$A x = b$, поскольку число условий больше числа неизвестных, лучшее что можно
сделать -- это минимизировать невязку $r = b - A x$. Задачи наименьших
квадратов формулируются как задачи минимизации евклидовой нормы невязки:

$$
||r||^{2}_{2} = \langle r, r \rangle \to min
$$

Один из способов решить задачу наименьших квадратов -- применить к ней прямой
и стандартный подход. Поскольку известно что $||r||^{2} = ||A x - b||^{2}$, то
производная по направлению $\partial x$ равна

$$
\nabla_{x} ||r||^{2} \cdot \partial x = 2 \langle A \partial x, b - A x\rangle =
2 \partial x^{T} (A^{T} b - A^{T} A x)
$$

Минимум

\section{<Временный раздел>}

Источники: \cite{BertsekasTsitsiklis1989, BierlaireTointTuyttens1991,
CantarellaPiatek2004, ChenDonohoSaunders2001, Golub1965, Lanczos1950,
PaigeSaunders1982, Saunders2013}.

\newpage
\bibliographystyle{plain}
\bibliography{references}

\end{document}
