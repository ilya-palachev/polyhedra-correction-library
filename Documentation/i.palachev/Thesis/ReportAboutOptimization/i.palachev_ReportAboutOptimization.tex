
% Copyright (c) 2009-20134 Ilya Palachev <iliyapalachev@gmail.com>
% This document contains the report about reconstruction of convex sets.

% Declare the class of document: size of paper, size of font, and etc.
% Type of the document is "article".
\documentclass[a4paper, 12pt, titlepage]{article} 

% Package that enables setting the size of free spaces at the border of the 
% page with the command  \geometry :
\usepackage{geometry}
\geometry {
   left=3cm,
   right=1.5cm,
   top=2cm,
   bottom=2cm
}

\usepackage[utf8]{inputenc}

\usepackage[english,russian]{babel}

\usepackage{amsmath}

%\usepackage{cmap}

\usepackage{indentfirst}

\usepackage{a4wide,amssymb}

%\usepackage[pdftex]{graphicx}

%\usepackage{wrapfig}

%\linespread{1.3}               % полтора интервала. Если 1.6, то два интервала
\pagestyle{plain}               % номерует страницы

\usepackage{graphicx}
\renewcommand{\topfraction}{1}
\renewcommand{\textfraction}{0}

% Package that enables usage of theorems and definition designed in a 
% standard way.
% http://en.wikibooks.org/wiki/LaTeX/Theorems
\usepackage{amsthm}

% Create environment for smart definitions
\theoremstyle{definition}
\newtheorem{SmartDefinition}{Определение}

% Create environment for smart theorems
\theoremstyle{plain}
\newtheorem{SmartTheorem}{Теорема}

% Create environment for smart lemmas
\theoremstyle{plain}
\newtheorem{SmartLemma}{Лемма}

% The following code enables back references:
\usepackage{color} 
\definecolor{darkgreen}{rgb}{0,.5,0} 
\usepackage[unicode,colorlinks,filecolor=blue,citecolor=darkgreen,pagebackref]
{hyperref}

% The package that provides symbols like \Square:
\usepackage{wasysym}

%opening
\title{Отчет по методам оптимизации}
\author{Палачев Илья}

\begin{document}

\maketitle

\tableofcontents

\newpage
\section{Линейное программирование}
Одним из самых распространенных классов задач оптимизации являются задачи
\textbf{линейного программирования}. Методы решения этих задач имеют широкое
приложение в разнообразных прикладных областях науки. Самым распространенным
методом решения задач линейного программирования является так называемый
\textbf{симплекс-метод}, который будет далее описан в этом разделе.

\textit{Общая задача линейного программирования} формулируется следующим
образом.

\begin{equation}
 J(u) = c_{1} u^{1} + \ldots + c_{n} u^{n} \to min
\end{equation}

при условиях

\begin{equation}
 u^{k} \geq 0, \;\;\;\;\; k \in I
\end{equation}

\begin{equation}
 \left\{
  \begin{aligned}
   a_{11} u^{1} + \ldots + a_{1n} u^{n} \leq 0 \\
   \ldots \\
   a_{n1} u^{1} + \ldots + a_{nn} u^{n} \leq 0 \\
  \end{aligned}
 \right.
\end{equation}

\begin{equation}
 \left\{
  \begin{aligned}
   a_{m1} u^{1} + \ldots + a_{mn} u^{n} = 0 \\
   \ldots \\
   a_{s1} u^{1} + \ldots + a_{sn} u^{n} = 0 \\
  \end{aligned}
 \right.
\end{equation}


\section{Рассматриваемые задачи}
\label{sec:problem}

В предыдущем отчете было указано, что задача восстановления многогранника по 
набору его теневых контуров сводится в том или ином виде к задаче квадратичной 
минимизации. В рассматриваемом случае функционал представлял из собой сумму 
квадратов разностей опорных чисел и их экспериментальных измерений:

\begin{equation}
\label{eq:problem1}
 I(h) = I(h_{1}, h_{2}, \ldots, h_{n}) = ||h - h^{0}||_{2}^{2} =
 \sum \limits_{i = 1}^{n} (h_{i} - h_{i}^{0})^{2} \to min
\end{equation}

где числа $h_{i}^{0}$ есть экспериментальные измерения величин $h_{i}$.

При этом требовалось, чтобы вектор опорных чисел удовлетворял условию

\begin{equation}
\label{eq:problem1-cond}
 Q h \geq 0
\end{equation}

где матрица $Q$ представляла собой матрицу коэффициентов линейных неравенств
следующего типа:

\begin{equation}
\left|\begin{array}{cc}
  h_{1} & u_{1}^{T} \\
  h_{2} & u_{2}^{T} \\
  h_{3} & u_{3}^{T} \\
  h_{4} & u_{4}^{T} \\
\end{array}\right|
  \left|\begin{array}{cc}
  1 & u_{1}^{T} \\
  1 & u_{2}^{T} \\
  1 & u_{3}^{T} \\
  1 & u_{4}^{T} \\
\end{array}\right|
\geq 0
\end{equation}

где $u_{i}^{T}$ -- известные опорные направления, а $h_{i}$ -- неизвестные
опорные числа. Таким образом, матрица $Q$ представляет собой $m \times
n$-матрицу, причем в каждой ее строке содержится не более 4 ненулевых элементов,
и число $m$ ее строк не превышает $12 n$ (в соответствии с теоремой Эйлера о
числе вершин, ребер и граней графа на сфере).

Задачу минимизации (\ref{eq:problem1}) - (\ref{eq:problem1-cond}) можно 
сформулировать несколько более удобным образом, как это описано, например, в 
книге \cite{BertsekasTsitsiklis1989}: оптимальное решение задачи
(\ref{eq:problem1}) можно найти как

\begin{equation}
\label{eq:problem2}
 h^{*} = h^{0} - Q^{T} u^{*}
\end{equation}

где $u^{*}$ есть решение задачи минимизации функционала

\begin{equation}
\label{eq:problem2-cond}
 I(u_{1}, u_{2}, \ldots, u_{m}) = ||Q^{T} u - h^{0}||_{2}^{2} \to min
\end{equation}

при условиях неотрицательности переменных:

\begin{equation}
 u \geq 0
\end{equation}

Данный отчет посвящен поиску методов, которые позволили бы решить задачи
(\ref{eq:problem1}) - (\ref{eq:problem1-cond}) или (\ref{eq:problem2}) -
(\ref{eq:problem2-cond}) наиболее эффективно.

\section{Метод нормальных уравнений}

Если говорить о задачах наименьших квадратов вообще, то их можно выделить в
особый тип задач минимизации. Предположим, что матрица $A \in \mathbb{R}^{m
\times n}$ и $m > n$. В общем случае невозможно точно решить уравнение
$A x = b$, поскольку число условий больше числа неизвестных, лучшее что можно
сделать -- это минимизировать невязку $r = b - A x$. Задачи наименьших
квадратов формулируются как задачи минимизации евклидовой нормы невязки:

$$
||r||^{2}_{2} = \langle r, r \rangle \to min
$$

Один из способов решить задачу наименьших квадратов -- применить к ней прямой
и стандартный подход. Поскольку известно что $||r||^{2} = ||A x - b||^{2}$, то
производная по направлению $\partial x$ равна

$$
\nabla_{x} ||r||^{2} \cdot \partial x = 2 \langle A \partial x, b - A x\rangle =
2 \partial x^{T} (A^{T} b - A^{T} A x)
$$

Минимум

\section{<Временный раздел>}

Источники: \cite{BertsekasTsitsiklis1989, BierlaireTointTuyttens1991,
CantarellaPiatek2004, ChenDonohoSaunders2001, Golub1965, Lanczos1950,
PaigeSaunders1982, Saunders2013}.

\newpage
\bibliographystyle{plain}
\bibliography{references}

\end{document}
