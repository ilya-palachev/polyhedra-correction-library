% Copyright (c) 2009-2015 Ilya Palachev <iliyapalachev@gmail.com>

\documentclass[a4paper, 10pt]{article}

\usepackage[utf8]{inputenc}
\usepackage[english,russian]{babel}

\title{О МЕТОДАХ ВОССТАНОВЛЕНИЯ ГЕОМЕТРИЧЕСКИХ ТЕЛ ПО ИЗМЕРЕНИЯМ ОПОРНОЙ ФУНКЦИИ
И ИХ ПРИЛОЖЕНИИ К ЗАДАЧЕ ВОССТАНОВЛЕНИЯ ТЕЛ ПО ТЕНЕВЫМ КОНТУРАМ}
\author{Палачев\,И.\,А.}

\begin{document}
\maketitle
\begin{abstract}
 Производится обзор методов восстановления геометрических тел (двумерных и
 трехмерных) по измерениям опорной функции. Рассматривается задача линейного
 программирования, лежащая в основе одного из этих методов. Доказывается
 достаточное условие избыточности ограничений задачи, позволяющее значительно
 уменьшить число ограничений. Рассматривается интерпретация теневых контуров
 как набора измерений опорной функции. Приводятся результаты проверки
 алгоритма на реальных данных, полученных из производственных теневых контуров.
\end{abstract}

\textbf{Ключевые слова:} опорная функция, восстановление геометрических тел,
линейное программирование, квадратичное программирование, теневой контур,
преобразование двойственности

\textbf{1. Введение.} Поводом для данного исследования послужила задача
восстановления трехмерных тел по теневым контурам, которая возникает при
построении моделей драгоценных камней в ювелирной промышленности. Теневые
контуры получаются с помощью фотографирования камня, стоящего на вращающейся
подставке. В результате обработки растрового изображения (фотографии) получается
теневой контур --- многоугольник, являющийся границей тени камня на некоторой
вертикальной плоскости.

При изучении данного вопроса было замечено его удивительное сходство с другими
задачами, имевшими совершенно иное прикладное происхождение и, как оказалось,
изучавшимися в иностранной научной литературе в течение последних почти 30 лет.
Ниже будет приведен краткий обзор этих задач и методов их решения.

Исследование преследует двоякую цель. Во-первых, показать что методы
восстановления геометрических тел по измерениям опорной функции применимы к
исходной прикладной задаче, т. е. что теневые контуры можно интерпретировать
как набор измерений опорной функции тела. Во-вторых, доказать свойство
одного из методов, позволяющее значительно ускорить алгоритм решения задачи.



\end{document}
